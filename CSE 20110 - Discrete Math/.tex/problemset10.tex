% Packages 
%--------------------

\documentclass[11pt,a4paper]{article}
\usepackage[utf8x]{inputenc}
\usepackage[T1]{fontenc}
\usepackage{amsthm,amsmath,amssymb}
\usepackage{mathrsfs}
\usepackage{xcolor}
\usepackage{siunitx} % Scientific notation

\usepackage[pdftex]{graphicx} % Required for including pictures
\usepackage[pdftex,linkcolor=black,pdfborder={0 0 0}]{hyperref} % Format links for pdf
\usepackage{calc} % To reset the counter in the document after title page
\usepackage{enumitem} % Includes lists
\usepackage{ stmaryrd }
\frenchspacing % No double spacing between sentences
\linespread{1.2} % Set linespace
\usepackage[a4paper, lmargin=0.13\paperwidth, rmargin=0.13\paperwidth, tmargin=0.1111\paperheight, bmargin=0.1111\paperheight]{geometry} %margins
\usepackage{parskip} % 2 enters makes a new line

\usepackage[all]{nowidow} % Tries to remove widows
\usepackage[protrusion=true,expansion=true]{microtype} % Improves typography, load after fontpackage is selected
%---------------------------

\begin{document} 
\title{Discrete Math Problem Set 10}
\author{Will Krzastek}
\date{April 22nd, 2024}
\maketitle

\begin{enumerate}

    % 1
    \item 
    Construct the explicit functions requested below. 
    \begin{enumerate}

        % 1a
        \item 
        A bijection from $\{x \in \mathbb{R} \mid -1 < x < 1 \} $ to $\{x \in \mathbb{R} \mid -\pi < x < \pi \} $. 

        \emph{\textbf{Proof.}} Let $A := \{x \in \mathbb{R} \mid -1 < x <1 \} $ and let $B := \{x \in \mathbb{R} \mid -\pi < x < \pi \} $. Let $f: A \rightarrow B$. We will define $f$ as $f(a) = a \cdot \pi $ where $a \in A $. Now, we will show that $f$ is a bijection. \\
        \emph{Injective:} \\
        Let $a, b \in A$. Assume $f(a) = f(b) $. Thus by the definition of $f$, $a \pi = b \pi $. So, $a = b$. \\
        Thus, $f$ is injective. \\
        \emph{Surjective:} \\
        Let $c \in B$. Let $d := \frac{c}{\pi} $. By the definition of $B$, we know that  $-\pi < c < \pi $, thus $-1 < d < 1 $. So, $d \in A $.
        Thus, we obtain $f(d) = d \pi = \frac{c}{\pi} \pi = c$. Because we chose an arbitrary element of $B$, we know that every element of $B$ has an element of the domain which maps to it. \\
        Thus, $f$ is surjective. \\
        Thus, $f$ is a bijection.
        \begin{flushright}
            $\blacksquare$
        \end{flushright}

        % 1b
        \item 
        A surjection from $\mathbb{N} $ to $\{p \mid p \text{ is prime} \} $. 

        \emph{\textbf{Proof.}} Let $P := \{p \mid p \text{ is prime} \} $. \\
        Let $f: \mathbb{N} \rightarrow P $ be defined as $f(n) := p$ such that $|\{q \in P \mid q \leq p \}| = n$. We will show that $f$ is a surjection. \\
        Let $k \in P $ where $k \in \mathbb{N} $. This means that $k$ is prime by definition of $P$. Observe, $k$ is finite. This lets us know that there are finitely many prime numbers less than $k$. As such, $|\{q \in P \mid q \leq k\}| $ is finite, which allows us to say that $|\{q \in P \mid q \leq k \}| = m $ for some $m \in \mathbb{N} $.
        Then by the definition of $f$, we know that $f(m) = k $. Because we showed this for an arbitrary element of $P$, we know that every element of $P$ has an element of the domain which maps to it. \\
        Therefore, $f$ is surjective.
        \begin{flushright}
            $\blacksquare$
        \end{flushright}    

        % 1c
        \item 
        An injection from $X$ to $\mathbb{P}(X) $ for every set $X$. 

        \emph{\textbf{Proof.}} Let $X$ be a set. \\
        Let $f: X \rightarrow \mathbb{P}(X) $ be defined as  $f(x) := \{x\} $. We will show that $f$ is injective. \\
        Let $a, b \in X$. Assume $f(a) = f(b)$. \\
        This means that $\{a\} = \{b\} $. By definition, we know $\forall c (c \in \{a\} \Longleftrightarrow c \in \{y\}) $. Therefore, all elements of $a$ and $b$ are the same. So, $a = b$.  \\
        Therefore, $f$ is injective.  
        \begin{flushright}
            $\blacksquare$
        \end{flushright}    
        
        % 1d
        \item 
        A surjection from $\mathbb{P}(X) $ to $X$ for every set $X \neq \varnothing $.   

        \emph{\textbf{Proof.}} Let $X$ be a set where $X \neq \varnothing$. Let $x \in X $.  \\
        Let $f: \mathbb{P}(X) \rightarrow X $ be defined 
        $$
        f(a) =
        \begin{cases}
            \cup a &\quad |a| = 1 \\
            x &\quad |a| \neq 1
        \end{cases}
        $$
        We will show that $f$ is surjective. \\
        Let $y \in X $. By the definition of a subset, we know that $\{y\} \subseteq X $, which means that $\{y\} \in \mathbb{P}(X) $ by the definition of a power set. We also know that $|\{y\}| = 1$ because there is only one element. 
        Therefore by the definition of $f$, we know that $f(\{y\}) = \cup \{y\} = y$. \\ 
        Because we showed this for an arbitrary element of $X$, we know that $f$ is surjective for every set $X$ where $X \neq \varnothing$.  
        \begin{flushright}
            $\blacksquare$
        \end{flushright}    

    \end{enumerate}

    % 2
    \item 
    Let $A$ be an arbitrary finite set of cardinality $|A| = n $, where $n \in \mathbb{N} $. How many finite strings over $A$ are there? 

    \emph{\textbf{Proof.}} Let $A$ be a set, where $|A| = n$ for some $n \in \mathbb{N} $. Let $s := \{ z \mid (\exists c \in \mathbb{N})(z = \{s \mid s: c \rightarrow A \}) \} $. Observe, $\cup s = \bigcup\limits_{i \in \mathbb{N}}^{\infty} \{s \mid s: i \rightarrow A \} $. 
    Let's define $B \in s $ for some $m \in \mathbb{N}$. So, $B = \{s \mid s: m \rightarrow A \} $. Thus, $|B| = |A|^m = n^m $ by \emph{Theorem 6.10}. So, $|B| $ is finite thus $B$ is countable. This means that $\cup s $ is a countable collection of countable sets. Therefore by \emph{Theorem 8.8}, we know that $|\cup s| $ is countable, meaning $|\cup s| \leq |\mathbb{N}| $. \\
    Now, we would like to prove that $\cup s $ is an infinite set. Towards a contradiction, assume $\cup s $ is finite. This means that for some $l \in \mathbb{N} $, $|\cup s| = l $. So, there exists some surjection $f: l \rightarrow \cup s $. \\ 
    Let $a \in A $. Let $h := f(0) +\!\!\!+ f(1) +\!\!\!+ \dots f(l - 1) +\!\!\!+ a $. So, $|h| = \displaystyle\sum_{i=0}^{l-1} |f(i)| + 1 $.
    So by definition, $(\forall k \in l)(|f(k)|) \leq \displaystyle\sum_{i=0}^{l-1} |f(i)| < \displaystyle\sum_{i=0}^{l-1} |f(i)| + 1 = |h|$.  
    This tells us that $(\forall k \in l)(|f(k)| \neq |h|) $. Therefore, $(\forall k \in l)(f(k) \neq h) $. So, $h$ is a finite string over $A$ where $h \in \cup s $, but $f$ does not map to $h$. Therefore $f$ is not a surjection. However, we assumed $f $ was a surjection $\lightning$. \\ 
    Therefore, $(\forall l \in \mathbb{N})(\cup s \neq l) $, so $\cup s $ is infinite. \\ 
    Therefore, we know that $\cup s $ is both countable and infinite. \\ 
    Therefore by definition, we know  $|\cup s| = |\mathbb{N}| = \aleph_0$. So, there are $\aleph_0 $ finite strings over $A$. 
    \begin{flushright}
        $\blacksquare$
    \end{flushright} 
    
    % 3
    \item 
    Imagine that, one day, you encounter a library.  At the entrance of this library is an enormous tome $\mathscr{B}$ listing \emph{all} of the \emph{possible} sentences in the English language, indexed by natural numbers.

    Walking past the entrance, you see that the library has rows of bookshelves numbered $0, 1, 2, \dots$, so that there is exactly one row of books for each $n \in \mathbb{N}$.
    A great owl---perched on the pedestal that supports $\mathscr{B}$---informs you that, for each $n \in \mathbb{N}$, the $n^{th}$ row of bookshelves contains \emph{all} of the books that could possibly ever be that begin with the $n^{th}$ sentence in $\mathscr{B}$.
    With the sound of granite scraping against marble, the doors to the library close behind you.
    The owl makes you the following proposal: you will be free to go \emph{if and only if} you can read every book in the library \emph{in a countable amount of time.}
    
    Will you be set free? The owl demands a proof to justify your answer. 

    \emph{\textbf{Proof.}} Observe a few things. There are countibly many sentences in  $\mathscr{B} $ because they are numbered.  $\mathscr{B} $ is a set of cardinality $\mathbb{N}  $ where $n \in \mathbb{N} $ is a sentence at the $n^{th}$ position of $\mathscr{B} $. We know that every book contains an infinitely countable amount of sentences, because they are indexed. Thus, we will represent the amount of sentences in each book with $|\mathbb{N}| $. Let  $S $ be $S := \{s \mid s: \mathbb{N} \rightarrow \mathbb{N} \} $ where $S$ is the set of all books of infinite length in the library.
    Thus, $S \subseteq  $ ``all of the books in the library''. \\
    Towards a contradiction, assume $|\mathbb{N}| \geq |S|  $. This means there exists some $f: \mathbb{N} \rightarrow S $ such that $f $ is surjective. Let $g \in S$. Consider, $g: \mathbb{N} \rightarrow \mathbb{N} $ where:
    $$
    g(i) :=
    \begin{cases}
        1 &\quad f(i)(i) \neq 1 \\
        2 &\quad f(i)(i) = 1
    \end{cases}
    $$
    Because  $f $ is surjective, $(\exists t \in \mathbb{N})(f(t) = g) $. So, $(\forall i \in \mathbb{N})(f(t)(i) = g(i)) $. However, observe $f(t)(t) = 1 \Leftrightarrow g(t) = 2 \neq 1 $, and $f(t)(t) \neq 1 \Leftrightarrow g(t) = 1 $.        
    This means that  $f(t)(t) \neq g(t) $, so $(\exists i \in \mathbb{N}) f(t)(i) \neq g(i) \lightning$. \\
    Therefore, $|\mathbb{N}| < |S| $. Because $|S| > |\mathbb{N}| $, we know $|S| > \aleph_0 $, which means by definition that $S $ is uncountibly infinite. So I will sadly \emph{not} be set free and the owl will own me forever :/
    \begin{flushright}
        $\blacksquare$
    \end{flushright} 

\end{enumerate}

\end{document}