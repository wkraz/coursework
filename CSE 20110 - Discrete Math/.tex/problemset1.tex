%--------------------
% Packages
% -------------------
\documentclass[11pt,a4paper]{article}
\usepackage[utf8x]{inputenc}
\usepackage[T1]{fontenc}
%\usepackage{gentium}
\usepackage{mathptmx} % Use Times Font
\usepackage{amsmath}

\usepackage[pdftex]{graphicx} % Required for including pictures
\usepackage[swedish]{babel} % Swedish translations
\usepackage[pdftex,linkcolor=black,pdfborder={0 0 0}]{hyperref} % Format links for pdf
\usepackage{calc} % To reset the counter in the document after title page
\usepackage{enumitem} % Includes lists

\frenchspacing % No double spacing between sentences
\linespread{1.2} % Set linespace
\usepackage[a4paper, lmargin=0.1666\paperwidth, rmargin=0.1666\paperwidth, tmargin=0.1111\paperheight, bmargin=0.1111\paperheight]{geometry} %margins
%\usepackage{parskip}

\usepackage[all]{nowidow} % Tries to remove widows
\usepackage[protrusion=true,expansion=true]{microtype} % Improves typography, load after fontpackage is selected

\usepackage{lipsum} % Used for inserting dummy 'Lorem ipsum' text into the template

%-----------------------
% Set pdf information and add title, fill in the fields
%-----------------------
\hypersetup{ 	
pdfsubject = {},
pdftitle = {},
pdfauthor = {}
}

%-----------------------
% Begin document
%-----------------------
\begin{document} 
\title{Discrete Math Problem Set 1}
\author{Will Krzastek}
\date{Sunday, January 28, 2024}
\maketitle
\begin{enumerate}
    \item[(50 points)~~~~1.]
    Determine the Truth Value of each sentence below.
    \begin{enumerate}
        \item 
        ``Madrid is the capital of Spain''
        
        $p := $ ``Madrid is the capital of Spain''. \\ Let $p \equiv \bot$. Here we disagree with $p$'s claim and assert that Madrid is \emph{not} the capital of Spain. Now I'm not from Spain, but Spain does not recognize a city other than Madrid as its capital, so it is impossible to find a counterexample for $p$'s assertion, and as such $p \not\equiv \bot$. You can also just type in ``Is Madrid the capital of Spain''? on Google and see that the answer is yes. Thus, $p \equiv \top$.
        \item 
        ``Santa Claus lives on the north pole.''

        $p := $ ``Santa Claus lives on the north pole''. \\ Again, let $p \equiv \bot$. We now believe that Santa Claus does \emph{not} live on the North Pole. To prove this asinine claim, we have to find out where Santa Claus lives. Unless they're being silenced, there is nobody in the world who can provide proof of residence for Santa Claus anywhere other than the North Pole. Thus, there is no example of $p$ being $\bot$, so $p \not\equiv \bot$. As such, $p \equiv \top$. 
      \item
        ``This sentence is \emph{false.}''

        $p := $ ``This sentence is \emph{false}''. \\ Consider $p \equiv \top$; By this logic, we agree with $p$ and must agree with its conclusion, which is that ``this sentence is false'', which reads as $p \equiv \bot$. So, when we take $p$ to $\equiv \top$, we reach the conclusion that $p$ actually $\equiv \bot$.  \\ Now consider $p \equiv \bot$; Here, we must disagree with $p$ and its conclusion: ``this sentence is false''. Thus, we take $p$ to say that this sentence is \emph{not} false. This reads as $p \not\equiv \bot$, which means that $p \equiv \top$ in this case. So, when we take $p$ to be $\equiv \bot$, we see that $p \equiv \top$. \\ As such, this sentence is a contradiction because it can be either $\top$ or $\bot$ depending on the initial assumption of $p$'s truth value, so this sentence possesses no truth value.  
      \item
        ``The set of all sets that don't contain themselves contains itself.''%

        $p := $ ``The set of all sets that don't contain themselves contains itself.'' \\ Let $p \equiv \top$. We agree with $p$ and believe that a set of every set that doesn't contain itself contains itself. If this is true and there is a set that contains itself in the set, then the set contains itself and the set no longer if full of sets that don't contain themselves. As such, we reach the conclusion that when we take $p$ to be $\equiv \top$, we see that $p \equiv \bot$. \\ Now let $p \equiv \bot$. We disagree with $p$ and believe that a set of every set that doesn't contain itself does not contain itself. This satisfies the conditions of the set of all sets, and as such when we let $p \equiv \bot$, we see that $p \equiv \top$. \\ Thus, this sentence is a paradox and contains no truth value because it cannot be true and false at the same time.
      \item
        ``Red is a beautiful color.''

        $p := $ ``Red is a beautiful color''. \\ Let $p \equiv \top$. We agree with $p$ and now know that red is a beautiful color. To demonstrate this to non-believers, we can take a look at a sunset, or any famous painting that includes red in it such as The Dessert: Harmony in Red by Matisse. I, and any sensible person, agree that these are beautiful images. In these images, red is also beautiful since it is a main reason why the images it is portrayed in are also beautiful. Thus, I find red to be beautiful. \\ Now let $p \equiv \bot$. Here we believe that red is not beautiful. However, none of red's qualities change from image to image since it is still the same color. And as such, red's beauty is constant no matter the context in which it is viewed. Thus, it is impossible to prove that red is not beautiful if you recognize that it is beautiful in beautiful images like sunsets, like I do. \\ I assert that red is beautiful in beautiful images that contain the color red, and none of red's qualities change since red is always the same red. As such, if red is beautiful at one point the same red will always be beautiful since red does not change, which means that this sentence is true. 
      \item
        ``Every declarative sentence is either \emph{true} or \emph{false} but not both.''

        $p := $ ``Every declarative sentence is either \emph{true} or \emph{false} but not both.'' \\ There are plenty of semantically nonsensical sentences like ``Colorless green ideas sleep furiously'' that are declarative and are neither true nor false. This sentence cannot be proven to be $\top$ or $\bot$, and as such it is in direct contrast with $p$'s claim that every declarative sentence is either true or false. Since this is a direct counterexample to $p$, $p \not\equiv \top$, and thus $p \equiv \bot$.
      \item
        ``If this sentence is \emph{false,} then $7$ is a prime number.''%

        $p := $ ``If this sentence is \emph{false,} then $7$ is a prime number.'' \\ This sentence is a conditional statement, and as such is represented by $p \rightarrow q$. \\ To show that ($p \rightarrow q) \equiv \top$, we can take $p$ to be $\equiv \top$, which gives us the conclusion that $p \equiv \bot$ and does not satisfy the antecedent. As such, the entire statement reads $top$ because the consequent is never reached and cannot be verifiably proven incorrect. \\ To show that ($p \rightarrow q) \equiv \bot$, we must take $p$ to $\equiv \bot$ to arrive at the conclusion that $p \equiv \top$ and satisfy the antecedent. Then we turn to $q$ and see that $q \equiv \top$ because $7$ is verifiably a prime number. Thus, $(p \rightarrow q) \equiv \top$ because we reach a $\top$ conclusion from the consequent. \\ And as such, there is no way to prove that $(p \rightarrow q) \equiv \bot$, so $(p \rightarrow q) \equiv \top$.
      \item
        ``The set of all sets contains itself.''

        $p := $ ``The set of all sets contains itself''. \\ A universal set that contains \emph{all} sets, must contain itself because it would break the definition of \emph{all} if it did not. If we take this statement to be false, we must find a counterexample set of all sets that does \emph{not} contain itself, and this is impossible because it would not contain \emph{all} sets then. \\ As such, this sentence $\not\equiv \bot$, and is then $\top$.
      \item
        ``This sentence is \emph{true.}''

        $p := $ ``This sentence is \emph{true}.'' \\ Take $p \equiv \top$. We then agree with $p$'s claim that this sentence is true. This agrees with the original assumption of $p$ being true, so $p \equiv \top$ if we take $p$ to be originally true. \\ Take $p \equiv \bot$. We now disagree with $p$'s claim and believe that this sentence is \emph{not} true, which means that the sentence is then false. This also agrees with the original assumption that $p$ is false. So, $p \equiv \bot$ if we take $p$ to be originally false. \\ As such, $p$ is a paradox, as it is both $\top$ and $\bot$, so $p$ possesses no truth value. 
      \item
        ``If this sentence is \emph{true,} then $2$ is an odd number.''%

        $p := $``'If this sentence is \emph{true,} then $2$ is an odd number.'' \\ This statement is in the form $p \rightarrow q$ since it is a conditional. \\ To show that ($p \rightarrow q) \equiv \top$, all we need to do is look at when $p \equiv \bot$. When $p \equiv \bot$, the antecedent is never satisfied, the consequent is never reached and therefore cannot be undoubtedly be proved $\bot$. As such, $p \rightarrow q$ cannot be proven $bot$, and therefore $(p \rightarrow q) \equiv \top$ when $p \equiv \bot$. \\ To show that $(p \rightarrow q) \equiv \bot$, we need to find a case when $p \equiv \top$ and $q \equiv \bot$, as this is the only case in which a conditional can be $\bot$. When $p \equiv \top$, the antecedent is satisfied so we turn to $q$ and read that ``$2$ is an odd number''. $2$ is in fact an even number, so $q \equiv \bot$ in this case. Now that $p \equiv \top$ and $q \equiv \bot$, ($p \rightarrow q) \equiv \bot$. Thus, when we take $p \equiv \top$, we find that $(p \rightarrow q) \equiv \bot$.  \\ As such, this statement is a paradox because it can be proven to be both $\top$ and $\bot$, so it possesses no truth value. 
    \end{enumerate}
    \item[(25 points)~~~~2.]
        Suppose we have an infinite sequence of sentences
        \begin{equation}
            S_0, S_1, S_2, \dots S_i, \dots
        \end{equation}
        where each sentence asserts that every sentence following it is \emph{false.}
        \begin{equation}
             S_i := \text{``}S_j \text{ is \emph{false} for all } j > i\text{.''}
        \end{equation}
        What are the truth values of the sentences in this sequence?

         If we take $S_0 \equiv \top$, we know that every sentence following $S_0$ is $\bot$, so every sentence up to $S_i$ is $\bot$. $S_1$ is then $\bot$ by this definition. If we take $S_1$ to be $\bot$, we disagree with its assertion that ``every sentence following it is \emph{false}''. Thus, there must be a $S_n$ ($n > 1)$ where $S_n \equiv \top$, which is in direct contrast with $S_0$'s that \emph{every} sentence is $\bot$. Thus, $S_0$ is an incorrect assumption when we take $S_0 \equiv \top$, since there is a counterexample at $S_1$ that disproves $S_0$'s assertion.
         
         Now take $S_0 \equiv \bot$. We know that every sentence following $S_0$ is \emph{not} $\bot$. This means there is an $S_k$ that $\equiv \top$, which means that $S_(k+1) \equiv \bot$, so we know there is an $S_j$ where $j > k + 1$ that is $\top$. Thus, $S_k$ is a contradiction because there is a counterexample at $S_(k+1)$. As such, $S_0$ is an incorrect assumption and must not be $\bot$ since there is a contradiction at $S_k$. 

         As such, $S_0$ contains no truth value because it is a contradiction in both cases. When $S_0$ is removed from $S_i$, the same contradiction occurs at $S_1$, and this cycle repeats for all of $S_i$. As a consequence, the entire series contains no truth value because every sentence is a contradiction. 
    \item[(25 points)~~~~3.]
        In the sentence below, ``\emph{you}'' refers to \emph{you}, the student reading these sentences and solving this problem set. Determine the truth value of the following sentence.
        \begin{equation}
            \text{``You have finitely many beliefs.''}
        \end{equation}

        $p := $ ``You have finitely many beliefs''  \\ 
        To confirm that you have finitely many beliefs, there must be an end to the beliefs, since finitude implies an eventual end. However, asserting ``I have finitely many beliefs'', is a belief in and of itself. If you call the amount of beliefs you previously had as $g$, you now possess $g + 1$ beliefs when you assert how many beliefs you have. When you think about having $g + 1$ beliefs, you now have $g + 2$ beliefs, and so on. This presents a loop that lasts forever, and eventually you realize that $g$ approaches $\infty$, so your beliefs are in fact infinite because you can never define a value for $g$ that does not increase when you think about it again. Therefore, $p \equiv \bot$.
\end{enumerate}
\end{document}
