% Packages 
%--------------------

\documentclass[11pt,a4paper]{article}
\usepackage[utf8x]{inputenc}
\usepackage[T1]{fontenc}
\usepackage{amsthm,amsmath,amssymb}
\usepackage{xcolor}
\usepackage{siunitx} % Scientific notation

\usepackage[pdftex]{graphicx} % Required for including pictures
\usepackage[pdftex,linkcolor=black,pdfborder={0 0 0}]{hyperref} % Format links for pdf
\usepackage{calc} % To reset the counter in the document after title page
\usepackage{enumitem} % Includes lists
\usepackage{ stmaryrd }
\frenchspacing % No double spacing between sentences
\linespread{1.2} % Set linespace
\usepackage[a4paper, lmargin=0.13\paperwidth, rmargin=0.13\paperwidth, tmargin=0.1111\paperheight, bmargin=0.1111\paperheight]{geometry} %margins
%\usepackage{parskip}

\usepackage[all]{nowidow} % Tries to remove widows
\usepackage[protrusion=true,expansion=true]{microtype} % Improves typography, load after fontpackage is selected
%---------------------------

\begin{document} 
\title{Discrete Math Problem Set 9}
\author{Will Krzastek}
\date{April 17th, 2024}
\maketitle

\begin{enumerate}
    
    % 1
    \item 
    A \emph{palindrome} of length $k \in \mathbb{N}$ over an alphabet $X$ is a string $S: k \rightarrow X $
    such that $(\forall i \in k)(s(i) = s(k - 1 - i)) $. How many possible palindromic words of length $k \in \mathbb{N} $ are there in the English lanugage? \bigskip \\
    \emph{\textbf{Proof.}} Let $X$ be the set of letters in the English language. Observe, $|X| = 26$. Let $k \in \mathbb{N}$.
    Let $A := \{s \mid s: k \rightarrow X \} $ where $(\forall i \in k)(s(i) = s(k - 1 - i)) $, where $A$ is the set of all functions that create palindromes of length $k$ out of the English alphabet. 
    Let $B := \{s \mid s: \lceil \frac{k}{2} \rceil \rightarrow X \} $, where $B$ is the set of all functions that create strings of length $\lceil \frac{k}{2} \rceil $ from the English alphabet. \\
    Now, we'll define $f: A \rightarrow B $ as $f(a) := b$, where $b: \lceil \frac{k}{2} \rceil \rightarrow X$, where $a(i) = b(i)  $ and $i \in \lceil \frac{k}{2} \rceil $. We want to show that $f$ is a bijection. First, we will show that $f$ is an injection. \\
    \emph{Injective:} \\
    Let $a_1, a_2 \in A $. Assume $f(a_1) = f(a_2) $. \\
    We want to show that $a_1 = a_2 $, so to do that we will show all inputs of each function lead to the same outputs: $(\forall i \in k)(a_1(i) = a_2(i)) $. There are two cases to observe: \\
    Case One, $i < \lceil \frac{k}{2} \rceil $: \\
    By our assumption, we know $f(a_1)(i) = f(a_2)(i) $. Then by the definition of $f$, we know $f(a_1)(i) = a_1(i) $ and $f(a_2)(i) = a_2(i) $. Therefore, we know $a_1(i) = a_2(i) $. \\
    Case Two, $i \geq \lceil \frac{k}{2} \rceil $: \\
    By our assumption, we know $f(a_1)(k - 1 - i) = f(a_2)(k - 1 - i) $. By definition, we then know $f(a_1)(k - 1 - i) = a_1(i) $ and $f(a_2)(k - 1 - i) = a_2(i) $. Therefore, we obtain $a_1(i) = a_2(i) $.  \\
    Therefore, $(\forall i \in k)(a_1(i) = a_2(i)) $. So, we know $a_1 = a_2 $. \\
    So, $f$ is injective. \\
    \emph{Surjective:} \\
    Let $b \in B$. Now, we will define $s: k \rightarrow X $ as $(\forall i \in k)$: 
    $$
    s(i) =
    \begin{cases}
      b(i) &\quad i < \lceil \frac{k}{2} \rceil \\
      b(k - 1 - i) &\quad i \geq \lceil \frac{k}{2} \rceil
    \end{cases}
    $$
    Observe, by the definition of $s$, $f(s) = b $. Thus, when $i < \lceil \frac{k}{2} \rceil $, $f(s) = s(i) = b(i) $. 
    Also observe, when $i \geq \lceil \frac{k}{2} \rceil $, we obtain $f(s) = s(i) = b(k - 1 - i) $. \\
    So, $(\forall b \in B)(\exists s \in A)(f(s) = b) $. \\
    So, $f$ is surjective. \\
    Therefore, $f$ is bijective. \\
    Therefore, $|A| = |B| $. By theorem 6.10, we know that $|B| = |X|^{\lceil \frac{k}{2} \rceil} $ because all the elements of $B$ are strings of length $\lceil \frac{k}{2} \rceil $ over the alphabet $X$. \\
    Because $|X| = 26$, we can then say $|B| = 26^{\lceil \frac{k}{2} \rceil} = |A|$. \\
    Thus, there are $26^{\lceil \frac{k}{2} \rceil} $ palindromes of length $k$ in the English language.
    \begin{flushright}
      $\blacksquare$
    \end{flushright}
    
    % 2
    \item 
    Given a natural number $k \in \mathbb{N} $, how many ordered pairs $(a, b) $ are there such that $a, b \in \mathbb{N} $ and $a + b = k$? \bigskip \\
    \emph{\textbf{Proof.}} Let $k \in \mathbb{N} $. We will define the set of all ordered pairs that sum to $k$ as $X := \{(a, b) \mid a, b \in \mathbb{N} \land a + b = k \} $. 
    Now, we will define a function $f: X \rightarrow k + 1 $ as $f((a, b)) = a $. Now, we will show that $f$ is a bijection. \\
    \emph{Injective:} \\
    Let $(a_1, b_1), (a_2, b_2) \in X $. Assume $f(a_1, b_1) = f(a_2, b_2) $. \\
    By the definition of $f$, we know $a_1 = a_2 $. By the definition of $X$, we know $b_1 = k - a_1 $, and $b_2 = k - a_2 $. Because $a_1 = a_2 $, we know $b_1 = b_2 $. \\
    Thus, $(a_1, b_1) = (a_2, b_2) $. \\
    So, $f$ is injective. \\
    \emph{Surjective:} \\
    Let $n \in k + 1 $. Now, consider the ordered pair $(n, k - n) $. We know $(n, k - n) \in X $ because $n, k - n \in \mathbb{N} \land n + k - n = k $. 
    Because we chose an arbitrary element of $k +1$ and showed an ordered pair exists in the domain that maps to it, we then know all elements of the codomain have an element in the domain which maps to it. \\
    Therefore, $f$ is surjective. \\
    So, $f$ is bijective. \\
    Thus, $|X| = |k + 1| $. We know $|k + 1| = k + 1 $, so $|X| = k + 1 $. Therefore, there are $k + 1$ ordered pairs $(a, b) $ such that $a, b \in \mathbb{N} \land a + b = k $.
    \begin{flushright}
      $\blacksquare$
    \end{flushright}

    % 3
    \item 
    Let $n \in \mathbb{N}_+$ and suppose you have an $n \times n$ chess board. We say two pieces on the board \emph{threaten each other} if it is possible for one to capture the other by moving to occupy its square on the next move. \bigskip \\
    In how many ways can $n \in \mathbb{N}_+ $ rooks possibly be arranged on an $n \times n$ chess board so that no two rooks threaten each other? \\
    \emph{\textbf{Proof.}} Let $n \in \mathbb{N}_+ $ and define the chess board as an $n_1 \times n_2 $ chessboard where $n_1 = n = n_2 $. 
    Let $R := \{1, 2, \dots n \} $ where $R$ is the set of rooks and $|R| = n$. Let $A$ be defined as all the possible ways the rooks can be arranged without threatening each other (unique row and column). \\
    Consider $f: n_1 \rightarrow n_2 $ where $f(i) = i $ and $i \in n_1, f(i) \in n_2$. This will be our process for placing rooks. \\
    We would like to prove that $|A| = n! $. To do so, we will use \emph{mathematical induction}. \\
    \emph{Basis step:} \\
    $n = 1$. This means that our $n_1 \times n_2 $ chess board is a $1 \times 1 $ chessboard. This means we have one possible choice for our rook. \\
    Observe, $f(1) = 1 $. So, we have placed the rook at $(1, 1) $. A rook cannot threaten itself and it is the only rook, so $|A| = 1 $. \\
    Recall, $1 = 1!$ by arithmetic.  \\
    \emph{Inductive step:} \\
    Inductive hypothesis: $|R| = n \implies |A| = n! $. \\
    We want to show that $|R| = n + 1 \implies |A| = (n + 1)! $. \\
    Observe, $n_1 = n + 1 = n_2 $ now. So, our chess board is an $(n + 1) \times (n + 1) $ chess board. \\
    Now, we will place one rook at an arbitrary row $m $, where $m \in n_1 $. \\
    Therefore, $f(m) = m $, which means that $m \in n_2 $ as well. So, we will place our rook at $(m, m) $. \\
    Because $m \in n_1, n_2$ and we placed the rook at $(m, m) $, we had $n + 1$ choices for where to place our rook, since $|n_1| = n + 1 $ and $m $ is an arbitrary element of $n_1$. \\
    Since $|R| = n + 1 $ and we just placed one rook, we now have $(n + 1) - 1 $ rooks to place, meaning $|R| = n $. \\
    By our Inductive Hypothesis, we know $|R| = n \implies |S| = n! $. \\
    Because we had $n + 1 $ choices for the first rook and $n! $ choices for the rest of the rooks, we then know that when $|R| = n +1 $, $|S| = (n + 1)n!$, which is equivalent to $(n + 1)! $ by arithmetic. \\
    Therefore, $|R| = n + 1 \implies |S| = (n + 1)! $. \\
    So, by \emph{mathematical induction}, we know that there are $n!$ ways to arrange $n$ rooks on an $n \times n $ chess board such that none of them threaten each other.     
    \begin{flushright}
      $\blacksquare$
    \end{flushright}
    
    % 4
    \item 
    We can write 4 as a sum of positive integers in the following ways. 
    \begin{equation*}
        \begin{tabular}{cccccccc}
                        &\quad& 1 + 1 + 2 &\quad& 1 + 3           \\
          1 + 1 + 1 + 1 &\quad& 1 + 2 + 1 &\quad& 2 + 2 &\quad& 4 \\
                        &\quad& 2 + 1 + 1 &\quad& 3 + 1           \\
        \end{tabular}
    \end{equation*}
    Given a natural number $n \in \mathbb{N} $, how many distinct ways are there to write $n$ as a sum of positive integers? \bigskip \\
    \emph{\textbf{Proof.}} Let $n \in \mathbb{N} $. \\
    If $n = 0 $: \\
    There are 0 ways that 2 positive integers can add to 0. \\
    If $n > 0 $: \\
    Consider $n = 1 + 1 + \dots 1 $ where the number of $1's = |n| $. Observe, there are $n - 1$ plus signs in this instance. \\
    Notice that there are at most $n - 1$ $+$ and $-$ signs in these equations, and there are only two choices: $+$, and $-$. With that being said, we want to represent the number of ways to write the sum of a positive integer using a binary string of length $n - 1$. \\
    To do so, we will create a function $f: n - 1 \rightarrow \{0, 1\} $. \\
    Let $i \in n - 1 $. \\
    We will define $f(i) := 0  $ if a $-$ sign is used, and $f(i) := 1 $ if a $+$ sign is used. \\
    Now, let $w := \{f \mid f: n - 1 \rightarrow \{0, 1\} \} $. \\
    Observe, $|\{0, 1\}| = 2 $, and $|n - 1| = n - 1 $. \\
    By Theorem 6.10, we then have $2^{n - 1} $ ways to write $n$ as a sum of positive integers.    
    \begin{flushright}
      $\blacksquare$
    \end{flushright}
    
    % 5
    \item 
    A hermetic monk in meditation is repeatedly ascending and descending a ladder with $n \in \mathbb{N}$ rungs. 
    The burdens of wisdom and devotion have left the monk's body frail, so he can only move by one or two rungs at-a-time. One day, as the monk ascends the ladder, a sudden tempest blows through his soul, and out of the whirlwind a fathomless voice calls out to him. \\
    \emph{``In how many ways can you do this?''} \\
    The monk ponders this question. After many years, he finds the answer, is enlightened, and immediately dies. What was the answer? \\
    (\emph{Hint:} Try strong induction. The scheme of strong induction is given below)
    \begin{equation*}
        \left(\varphi(0) \land
          \left(\forall k \in \mathbb{N}\right)
          \left(\left(\forall \ell \in \mathbb{N}\right)
            \left(\ell \leq k \Rightarrow \varphi\left(\ell\right)\right)
        \Rightarrow \varphi(k + 1)\right)\right)
        \Rightarrow \left(\forall n \in \mathbb{N}\right)\left(\varphi(n)\right)
    \end{equation*} 
    \emph{\textbf{Proof.}} 
    Let $n \in \mathbb{N} $. Let $A_n := \{f \mid (\exists k \in \mathbb{N}_+)(f: k \rightarrow \{1, 2\} \land (\sum_{i=0}^{k}s(i) = n)) \} $. 
    $A_n$ represents the set of all ways the monk can climb a ladder of $n$ rungs such that he only takes 1 or 2 steps at a time. We will represent $|A_n|  $ with a function $\varphi(n)  $. 
    To show that $\varphi(n) = \mathcal{F}(n + 1) $, we will use \emph{strong induction}. \\
    Also, recall the definition of the Fibonacci Sequence: \\ 
    $\mathcal{F}(0) = 0; \mathcal{F}(1) = 1; \mathcal{F}(n + 2) = \mathcal{F}(n + 1) + \mathcal{F}(n) $.  \\
    \emph{Basis steps:} \\
    $n = 0$: \\
    Observe that when $k = 0$, we obtain $f: 0 \rightarrow \{1, 2\} \land \sum_{i=0}^{0}s(i) = 0$. We know $\sum_{i=0}^{0}s(i) = 0  $ by the definition of summation,
    so $|A_0| = 1 $ because there only exists one valid $f$, and it occurs with the value $k = 0$. Thus, $\varphi(0) = 1 $, which means there's 1 way to climb a ladder with 0 rungs. Also observe, $\mathcal{F}(0 + 1) = 1 $.   \\
    $ n = 1$: \\
    Observe that when $k = 1$, we obtain $f: 1 \rightarrow \{1, 2\} \land \sum_{i=0}^{1}s(i) = 1 $. There is only one valid $f$, because we must have a nonzero value as our summation, since $n = 1$, and that only occurs when $k = 1$. 
    So, $|A_1| = 1 $, which means $\varphi(1) = 1 = \mathcal{F}(1 + 1) $. This means there is only one way to climb a ladder with one rung, which makes sense. \\
    $n = 2$: \\
    Observe that we must find all the $k$ values where $f: k \rightarrow \{1, 2\} \land \sum_{i=0}^{k}s(i) = 2  $. The only possible $k$ values are 1 and 2. \\
    If $k = 1$, we will only have one value of $i$, so the step will be of 2. If $k = 2$, we will have two values of $i$, so the steps will be of 1. Therefore, there are two valid values of $k$, which gives us 2 valid $f$'s. \\
    This means that $|A_2| = 2 = \varphi(2) = \mathcal{F}(2 + 1)$. This means that there are two ways to climb a ladder with 2 rungs: either by stepping by 1 twice, or by stepping by 2 once. \\
    \emph{Inductive Step:} \\
    Because we are using \emph{strong induction}, assume $(\forall \ell \in \mathbb{N})(\ell \leq n \implies \varphi(\ell) = \mathcal{F}(\ell + 1)) $. 
    This means that $\varphi(n) = \mathcal{F}(n + 1) = |A_n| $, and $\varphi(n - 1) = \mathcal{F}(n) = |A_{n-1}| $. We need to show that $\varphi(n + 1) = \mathcal{F}(n + 2) = |A_{n + 1}|$.
    To do that, we will show that $|A_{n+1}| = |A_n| + |A_{n-1}| $, because $\mathcal{F}(n+2) := \mathcal{F}(n+1) + \mathcal{F}(n) $. \\
    Now, we will define two subsets of $A_{n+1}$. \\
    Let $B_{n+1} \subseteq A_{n+1} $ such that $(\forall s \in B_{n+1})(s(0) = 1) $, where $B_{n+1}$ is the set of all ways to get up the ladder with the monk's first step being of length 1. \\
    Let $C_{n+1} \subseteq A_{n+1} $ such that $(\forall s \in C_{n+1})(s(0) = 2) $, where $C_{n+1} $ is the set of all ways to get up the ladder with the monk's first step being of length 2. \\
    Because there are only 2 choices for the monk's first step (1, 2), we know that $B_{n+1} \cup C_{n+1} = A_{n+1} $. We also know that $B_{n+1} \cap C_{n+1} = \varnothing$ because $(\forall s \in A_{n+1})(s(0) = 1 \Longleftrightarrow s(0) \neq 2$), so they share no elements. 
    Therefore by the \emph{Inclusion/Exclusion Principle}, we know $|A_{n+1}| = |B_{n+1}| + |C_{n+1}| $. 
    Now, we will show that $|B_{n+1}| = |A_n| $ and $|C_{n+1}| = |A_{n-1}| $. \\
    First, we will prove $|B_{n+1}| = |A_n| $: \\
    Let $f: B_{n+1} \rightarrow A_n $ be defined as $f(b) := a: k \setminus 1 \rightarrow \{1,2\}$ such that $b(i) = a(i) $ for all $i \in k \setminus 1 $. \\
    \emph{Injectivity}: \\
    Let $b_1, b_2 \in B_{n+1} $. Assume $f(b_1) = f(b_2) $. We will show that $(\forall i \in k)(b_1(k) = b_2(k)) $. \\
    If $i = 0$, we know that $b_1(0) = 1 = b_2(0) $ by the definition of $B_{n+1} $. So, $b_1(i) = b_2(i) $. \\
    If $i \neq 0 $, we know that $b_1(i) = f(b_1)(i -1) $ and $b_2(i) = f(b_2)(i - 1) $ by the definition of $f$. Because $f(b_1) = f(b_2) $, we obtain $f(b_1)(i - 1) = f(b_2)(i  -1) $, so we know $b_1 = b_2 $. So, $f$ is injective. \\
    \emph{Surjectivity:} \\
    Let $a \in A_n$. Let $s := (\forall i \in a)(s(i + 1) = a(i))$, and $s(0) = 1$. 
    By the definition of $A_n $, we know $\sum_{i=0}^{k} a(i) = n $. We also know that $\sum_{i=0}^{k} s(i) = 1 + \sum_{i=0}^{k} a(i) = n + 1 $. So, we know that $s \in B_{n+1} $, so there $a$ has an element of the domain that maps to it. Therefore $f$ is surjective. \\
    Thus, $f$ is bijective and $|B_{n+1}| = |A_n| $. \\
    Now, we will prove $|C_{n+1}| = |A_{n-1}| $: \\
    Let $g: C_{n+1} \rightarrow A_{n-1} $ be defined as $g(c) := a: k \setminus 1 \rightarrow \{1,2\}$ such that $c(i) = a(i) $ for all $i \in k \setminus 1 $. \\
    \emph{Injectivity}: \\
    Let $c_1, c_2 \in C_{n+1} $. Assume $g(c_1) = g(c_2) $. We will show that $(\forall i \in k)(c_1(k) = c_2(k)) $. \\
    If $i = 0$, we know that $c_1(0) = 2 = c_2(0) $ by the definition of $C_{n+1} $. So, $c_1(i) = c_2(i) $. \\
    If $i \neq 0 $, we know that $c_1(i) = g(c_1)(i -1) $ and $c_2(i) = g(c_2)(i - 1) $ by the definition of $g$. Because $g(c_1) = g(c_2) $, we obtain $g(c_1)(i - 1) = g(c_2)(i  -1) $, so we know $c_1 = c_2 $. So, $g$ is injective. \\
    \emph{Surjectivity:} \\
    Let $a \in A_{n-1}$. Let $s := (\forall i \in a)(s(i + 1) = a(i))$, and $s(0) = 2$. 
    By the definition of $A_{n-1} $, we know $\sum_{i=0}^{k} a(i) = n-1 $. We also know that $\sum_{i=0}^{k} s(i) = 1 + \sum_{i=0}^{k} a(i) = n $. So, we know that $s \in C_{n+1} $, so $a$ has an element of the domain that maps to it. Therefore $g$ is surjective. \\
    Thus, $g$ is bijective and $|C_{n+1}| = |A_{n-1}| $. \\
    So, $|B_{n+1}| = |A_n|$ and $|C_{n+1}| = |A_{n-1}| $. So, $|A_{n+1} | = |A_n| + |A_{n-1}|$. Thus, $\varphi(n+1) = \varphi(n) + \varphi(n-1) $. 
    By definition, we then know that $\varphi(n+1) = \mathcal{F}(n+1) + \mathcal{F}(n) $. By the definition of the Fibonacci Sequence, we then know that $\varphi(n+1) = \mathcal{F}(n+2)$. \\
    So, by \emph{strong induction}, $(\forall n \in \mathbb{N})(\varphi(n) = \mathcal{F}(n+1)) $. So, there are $\mathcal{F}(n+1) $ ways for the monk to climb a ladder with  $n$ rungs. 
    \begin{flushright}
      $\blacksquare$
    \end{flushright}

\end{enumerate}


\end{document}