%--------------------
% Packages
% -------------------
\documentclass[11pt,a4paper]{article}
\usepackage[utf8x]{inputenc}
\usepackage[T1]{fontenc}
%\usepackage{gentium}
\usepackage{mathptmx} % Use Times Font
\usepackage{amsthm,amsmath,amssymb}
\usepackage{xcolor}

\usepackage[pdftex]{graphicx} % Required for including pictures
\usepackage[swedish]{babel} % Swedish translations
\usepackage[pdftex,linkcolor=black,pdfborder={0 0 0}]{hyperref} % Format links for pdf
\usepackage{calc} % To reset the counter in the document after title page
\usepackage{enumitem} % Includes lists

\frenchspacing % No double spacing between sentences
\linespread{1.2} % Set linespace
\usepackage[a4paper, lmargin=0.1666\paperwidth, rmargin=0.1666\paperwidth, tmargin=0.1111\paperheight, bmargin=0.1111\paperheight]{geometry} %margins
%\usepackage{parskip}

\usepackage[all]{nowidow} % Tries to remove widows
\usepackage[protrusion=true,expansion=true]{microtype} % Improves typography, load after fontpackage is selected

%-----------------------
% Set pdf information and add title, fill in the fields
%-----------------------
\hypersetup{ 	
pdfsubject = {},
pdftitle = {},
pdfauthor = {}
}

%-----------------------
% Begin document
%-----------------------
\begin{document} %All text i dokumentet hamnar mellan dessa taggar, allt ovanför är formatering av dokumentet
\title{Discrete Math Problem Set 2}
\author{Will Krzastek}
\date{February 4, 2024}
\maketitle
We say that a proposition is a \emph{\color{red}{tautology}} if it is logically equivalent to $\top$ under any assignment of truth values to its variables.
\begin{enumerate}
    \item[(5 points) \quad 1.]
    Consider the following proof of $p \rightarrow (q \rightarrow r) \equiv (p \rightarrow q) \rightarrow r$.
    
    \emph{\textbf{Proof.}} Observe the following chain of reasoning.
    \begin{align*}
    p \rightarrow (q \rightarrow r) 
        &\equiv p \lor \neg(q \rightarrow r)   
            &\quad 
            &\text{by \emph{conditional disintegration}} \\ 
        &\equiv p \lor \neg(q \lor \neg r)
            &\quad
            &\text{by \emph{conditional disintegration}} \\
        &\equiv p \lor \neg q \lor \neg r
            &\quad
            &\text{by \emph{associativity}} \\
        &\equiv (p \lor \neg q) \lor \neg r
            &\quad
            &\text{by \emph{associativity}} \\
        &\equiv (p \rightarrow q) \lor \neg r
            &\quad
            &\text{by \emph{conditional disintegration}} \\
        &\equiv (p \rightarrow q) \rightarrow r
            &\quad
            &\text{by \emph{conditional disintegration}}
    \end{align*}
    Therefore, $p \rightarrow (q \rightarrow r) \equiv (p \rightarrow q) \rightarrow r$.

    Find all of the mistakes, if any, in this proof and \emph{explain why}.

    In steps 1, 2, 5, and 6 of the proof, \emph{conditional disintegration} is applied incorrectly. Conditional disintegration asserts that $p \rightarrow q \equiv \neg p \lor q$, but in all of these steps it is applied as $p \rightarrow q \equiv p \lor \neg q$ (in this case $q$ is $q \rightarrow r$)

    In steps 3, and 4 of the proof, \emph{associativity} is applied incorrectly. In step 3, the parentheses on $q \lor \neg r$ are removed, and then added on in step 4. Associativity means that the parentheses should have been moved in the same step, as you are not allowed to remove them or add them, only move them. In step 3 however, \emph{De Morgan's Laws} should have been used, as there is a negation preventing the parentheses from being associated. 
    \item[(40 points) \quad 2.]
    Prove the claims below \emph{without truth tables} for all propositions $p, q, r$.
    \begin{enumerate}
        \item 
        $p \rightarrow q \equiv \neg q \rightarrow \neg p$. \\
        \emph{\textbf{Proof.}} Observe the following chain of reasoning. \\ 
        Let $p$ and $q$ be arbitrary propositions.
    \begin{align*}
    p \rightarrow q
        &\equiv \neg p \lor q
            &\quad
            &\text{by \emph{conditional disintegration}} \\
        &\equiv q \lor \neg p
            &\quad
            &\text{by \emph{commutativity}} \\
        &\equiv \neg \neg q \lor \neg p
            &\quad
            &\text{by \emph{Double Negation}} \\
        &\equiv \neg q \rightarrow \neg p
            &\quad
            &\text{by \emph{conditional disintegration}}
    \end{align*}
    Therefore, $p \rightarrow q \equiv \neg q \rightarrow \neg p$. \\
        \begin{flushright}
            \textbf{Ergo.}
        \end{flushright}
        \item 
        $(p \land (p \rightarrow q)) \rightarrow q$ is a tautology. \\
        \emph{\textbf{Proof.}} Observe the following chain of reasoning. \\ 
        Let $p$ and $q$ be arbitrary propositions.
    \begin{align*}
    (p \land (p \rightarrow q)) \rightarrow q
        &\equiv \neg(p \land (p \rightarrow q)) \lor q
            &\quad
            &\text{by \emph{conditional disintegration}} \\
        &\equiv (\neg p \lor \neg(p \rightarrow q)) \lor q
            &\quad
            &\text{by \emph{De Morgan's Laws}} \\
        &\equiv (\neg p \lor \neg(\neg p \lor q)) \lor q 
            &\quad
            &\text{by \emph{conditional disintegration}} \\
        &\equiv (\neg p \lor (\neg \neg p \land \neg q)) \lor q
            &\quad
            &\text{by \emph{De Morgan's Laws}} \\
        &\equiv (\neg p \lor (p \land \neg q)) \lor q
            &\quad
            &\text{by \emph{Double Negation}} \\
        &\equiv ((\neg p \lor p) \land(\neg p \lor \neg q) \lor q
            &\quad
            &\text{by \emph{distributivity}} \\
        &\equiv (\top \land (\neg p \lor \neg q) \lor q
            &\quad
            &\text{by \emph{complement}} \\
        &\equiv (\neg p \lor \neg q) \lor q
            &\quad
            &\text{by \emph{identity}} \\
        &\equiv \neg p \lor (\neg q \lor q)
            &\quad
            &\text{by \emph{associativity}} \\
        &\equiv \neg p \lor \top
            &\quad
            &\text{by \emph{complement}} \\
        &\equiv \top
            &\quad
            &\text{by \emph{domination}}
    \end{align*}
    Therefore, $(p \land (p \rightarrow q)) \rightarrow q \equiv \top$ for all values of $p$ and $q$, and is a tautology.
        \begin{flushright}
            \textbf{Ergo.}
        \end{flushright}
        \pagebreak
        \item 
        $(\neg q \land (p \rightarrow q)) \rightarrow \neg p$ is a tautology. \\
        \emph{\textbf{Proof.}} Observe the following chain of reasoning. \\ 
        Let $p$ and $q$ be arbitrary propositions.
    \begin{align*}
    (\neg q \land (p \rightarrow q)) \rightarrow \neg p
        &\equiv (\neg q \land (\neg p \lor q)) \rightarrow \neg p 
            &\quad
            &\text{by \emph{conditional distribution}} \\
        &\equiv (\neg q \land (q \lor \neg p)) \rightarrow \neg p
            &\quad
            &\text{by \emph{commutativity}} \\
        &\equiv ((\neg q \land q) \lor (\neg q \land \neg p)) \rightarrow \neg p
            &\quad
            &\text{by \emph{distributivity}} \\
        &\equiv (\bot \lor (\neg q \land \neg p)) \rightarrow \neg p
            &\quad 
            &\text{by \emph{complement}} \\
        &\equiv (\neg q \land \neg p) \rightarrow \neg p
            &\quad
            &\text{by \emph{identity}} \\
        &\equiv \neg(\neg q \land \neg p) \lor \neg p
            &\quad 
            &\text{by \emph{conditional disintegration}} \\
        &\equiv (\neg \neg q \lor \neg \neg p) \lor \neg p
            &\quad
            &\text{by \emph{De Morgan's Laws}} \\
        &\equiv (q \lor p) \lor \neg p
            &\quad
            &\text{by \emph{Double Negation}} \\
        &\equiv q \lor (p \lor \neg p)
            &\quad 
            &\text{by \emph{associativity}} \\
        &\equiv q \lor \top
            &\quad
            &\text{by \emph{complement}} \\
        &\equiv \top
            &\quad
            &\text{by \emph{domination}}
    \end{align*}
    Therefore, $(\neg q \land (p \rightarrow q)) \rightarrow \neg p \equiv \top$, and is a tautology.
        \begin{flushright}
            \textbf{Ergo.}
        \end{flushright}
        \pagebreak
        \item 
        $(p \rightarrow q) \rightarrow ((p \rightarrow \neg q) \rightarrow \neg p)$ is a tautology. 
        \emph{\textbf{Proof.}} Observe the following chain of reasoning. \\ 
        Let $p$ and $q$ be arbitrary propositions.
    \begin{align*}
    (p \rightarrow q) \rightarrow ((p \rightarrow \neg q) \rightarrow \neg p)
        &\equiv (p \rightarrow q) \rightarrow ((\neg p \lor \neg q) \rightarrow \neg p)
            &\quad
            &\text{by \emph{conditional disintegration}} \\
        &\equiv (p \rightarrow q) \rightarrow (\neg(\neg p \lor \neg q) \lor \neg p)
            &\quad
            &\text{by \emph{conditional disintegration}} \\
        &\equiv (p \rightarrow q) \rightarrow ((\neg \neg p \land \neg \neg q) \lor \neg p)
            &\quad
            &\text{by \emph{De Morgan's Rules}} \\
        &\equiv (p \rightarrow q) \rightarrow ((p \land q) \lor \neg p)
            &\quad
            &\text{by \emph{double negation}} \\
        &\equiv (p \rightarrow q) \rightarrow (\neg p \lor (p \land q))
            &\quad
            &\text{by \emph{commutativity}} \\
        &\equiv (p \rightarrow q) \rightarrow ((\neg p \lor p) \land (\neg p \lor q))
            &\quad
            &\text{by \emph{distributivity}} \\
        &\equiv (p \rightarrow q) \rightarrow (\top \land (\neg p \lor q))
            &\quad
            &\text{by \emph{complement}} \\
        &\equiv (p \rightarrow q) \rightarrow (\neg p \lor q)
            &\quad
            &\text{by \emph{identity}} \\
        &\equiv (p \rightarrow q) \rightarrow (p \rightarrow q)
            &\quad
            &\text{by \emph{conditional disintegration}} \\
    \end{align*}
    Now let $s$ be a proposition $\equiv (p \rightarrow q)$. Therefore we can rewrite $(p \rightarrow q) \rightarrow (p \rightarrow q)$ as $s \rightarrow s$. $s \rightarrow s \equiv \top$, as proven in 3A, so therefore $(p \rightarrow q) \rightarrow (p \rightarrow q) \equiv \top$. As such, $(p \rightarrow q) \rightarrow ((p \rightarrow \neg q) \rightarrow \neg p) \equiv \top$, and is a tautology. 
        \begin{flushright}
            \textbf{Ergo.}
        \end{flushright}
    \end{enumerate}
    \item[(40 points) \quad 3.]
    In this problem, we will progressively establish that the alternative axioms Hilbert proposed are all tautologies \emph{without truth tables}. Here, the variables $p, q, r$ all represent arbitrary propositions.
    \begin{enumerate}
        \item
        Show $p \rightarrow p$ is a tautology. \\
        \emph{\textbf{Proof.}} Observe the following chain of reasoning. \\ 
        Let $p$ be an arbitrary proposition.
        \begin{align*}
        (p \rightarrow p)
            &\equiv (\neg p \lor p)
                &\quad
                &\text{by \emph{conditional disintegration}} \\
            &\equiv \top
                &\quad
                &\text{by \emph{complement}}
        \end{align*}
        Therefore, ($p \rightarrow p) \equiv \top$, and is a tautology.
        \begin{flushright}
            \textbf{Ergo.}
        \end{flushright}
        \pagebreak
        \item 
        Show $(p \rightarrow q) \rightarrow (\neg q \rightarrow \neg p)$ is a tautology. \\
        \emph{\textbf{Proof.}} Observe the following chain of reasoning. \\ 
        Let $p, q$ be arbitrary propositions.
        \begin{align*}
        (p \rightarrow q) \rightarrow (\neg q \rightarrow \neg p)
            &\equiv (p \rightarrow q) \rightarrow (p \rightarrow q)
                &\quad
                &\text{as proven in 2A} \\
        \end{align*}
        Now let $r$ be a proposition $\equiv (p \rightarrow q)$. We can then rewrite $(p \rightarrow q) \rightarrow (p \rightarrow q)$ as $r \rightarrow r$. As proven in 3A, we know that $r \rightarrow r \equiv \top$, so $(p \rightarrow q) \rightarrow (p \rightarrow q) \equiv \top$. As such, $(p \rightarrow q) \rightarrow (\neg q \rightarrow \neg p) \equiv \top$ and is a tautology.
        \begin{flushright}
            \textbf{Ergo.}
        \end{flushright}
        \item 
        Show $p \rightarrow (q \rightarrow p)$ is a tautology. \\
        \emph{\textbf{Proof.}} Observe the following chain of reasoning. \\ 
        Let $p, q$ be arbitrary propositions.
        \begin{align*}
        p \rightarrow (q \rightarrow p)
            &\equiv \neg p \lor (q \rightarrow p)
                &\quad
                &\text{by \emph{conditional disintegration}} \\
            &\equiv \neg p \lor (\neg q \lor p)
                &\quad
                &\text{by \emph{conditional disintegration}} \\
            &\equiv \neg p \lor (p \lor \neg q)
                &\quad 
                &\text{by \emph{commutativity}} \\
            &\equiv (\neg p \lor p) \lor \neg q
                &\quad
                &\text{by \emph{associativity}} \\
            &\equiv \top \lor \neg q
                &\quad 
                &\text{by \emph{complement}} \\
            &\equiv \top
                &\quad 
                &\text{by \emph{domination}}
        \end{align*}
        Therefore, $p \rightarrow (q \rightarrow p) \equiv \top$, and is a tautology.
        \begin{flushright}
            \textbf{Ergo.}
        \end{flushright}
        \pagebreak
        \item 
        Show $(p \rightarrow (q \rightarrow r)) \rightarrow ((p \rightarrow q) \rightarrow (p \rightarrow r))$ is a tautology. \\
        \emph{\textbf{Proof.}} Observe the following chain of reasoning. \\ 
        Let $p, q, r$ be arbitrary propositions.
        
        $(p \rightarrow (q \rightarrow r)) \rightarrow ((p \rightarrow q) \rightarrow (p \rightarrow r))$
        \begin{align*}
            &\equiv (p \rightarrow (q \rightarrow r)) \rightarrow ((\neg p \lor q) \rightarrow (p \rightarrow r)) 
                &\quad
                &\text{by \emph{conditional disintegration}} \\
            &\equiv (p \rightarrow (q \rightarrow r)) \rightarrow ((\neg p \lor q) \rightarrow (\neg p \lor r)) 
                &\quad
                &\text{by \emph{conditional disintegration}} \\
            &\equiv (p \rightarrow (q \rightarrow r)) \rightarrow (\neg (\neg p \lor q) \lor (\neg p \lor r))
                &\quad
                &\text{by \emph{conditional disintegration}} \\
            &\equiv (p \rightarrow (q \rightarrow r)) \rightarrow ((\neg \neg p \land \neg q) \lor (\neg p \lor r))
                &\quad
                &\text{by \emph{De Morgan's Rules}} \\
            &\equiv (p \rightarrow (q \rightarrow r)) \rightarrow ((p \land \neg q) \lor (\neg p \lor r))
                &\quad
                &\text{by \emph{double negation}} \\
            &\equiv (p \rightarrow (q \rightarrow r)) \rightarrow ((\neg p \lor r) \lor (p \land \neg q))
                &\quad
                &\text{by \emph{commutativity}} \\
            &\equiv (p \rightarrow (q \rightarrow r)) \rightarrow ((((\neg p \lor r) \lor p) \land ((\neg p \lor r) \lor \neg q))
                &\quad
                &\text{by \emph{distributivity}} \\
            &\equiv (p \rightarrow (q \rightarrow r)) \rightarrow ((((\neg p \lor p) \lor r) \land ((\neg p \lor r) \lor \neg q))
                &\quad
                &\text{by \emph{associativity}} \\
            &\equiv (p \rightarrow (q \rightarrow r)) \rightarrow (((\top \lor r) \land ((\neg p \lor r) \lor \neg q))
                &\quad
                &\text{by \emph{complement}} \\
            &\equiv (p \rightarrow (q \rightarrow r)) \rightarrow ((\top \land ((\neg p \lor r) \lor \neg q))
                &\quad 
                &\text{by \emph{domination}} \\
            &\equiv (p \rightarrow (q \rightarrow r)) \rightarrow ((\neg p \lor r) \lor \neg q))
                &\quad
                &\text{by \emph{identity}} \\
            &\equiv (p \rightarrow (q \rightarrow r)) \rightarrow ((\neg p \lor (\neg q \lor r))
                &\quad
                &\text{by \emph{associativity + commutativity}} \\
            &\equiv (p \rightarrow (q \rightarrow r)) \rightarrow (( \neg \neg p \rightarrow (\neg q \lor r))) 
                &\quad
                &\text{by \emph{conditional disintegration}} \\
            &\equiv (p \rightarrow (q \rightarrow r)) \rightarrow ((p \rightarrow (\neg q \lor r)))
                &\quad
                &\text{by \emph{double negation}} \\
            &\equiv (p \rightarrow (q \rightarrow r)) \rightarrow ((p \rightarrow (\neg \neg q \rightarrow r)))
                &\quad
                &\text{by \emph{conditional disintegration}} \\
            &\equiv (p \rightarrow (q \rightarrow r)) \rightarrow ((p \rightarrow (q \rightarrow r)))
                &\quad
                &\text{by \emph{double negation}}
        \end{align*}
        Now let $s$ be a proposition that is $\equiv (p \rightarrow (q \rightarrow r))$. We can now rewrite $(p \rightarrow (q \rightarrow r)) \rightarrow ((p \rightarrow (q \rightarrow r)))$ as $s \rightarrow s$. As proven in 3A, we know that $s \rightarrow s \equiv \top$, so therefore  $(p \rightarrow (q \rightarrow r)) \rightarrow ((p \rightarrow (q \rightarrow r))) \equiv \top$, and $(p \rightarrow (q \rightarrow r)) \rightarrow ((p \rightarrow q) \rightarrow (p \rightarrow r))$ is a tautology.
        \begin{flushright}
            \textbf{Ergo.}
        \end{flushright}
    \end{enumerate}
    \item[(10 points) \quad 4.]
    Show that $\neg$ and $\land$ are sufficient to express \emph{any} proposition. \\
    \emph{\textbf{Proof.}} Observe the following chain of reasoning. \\
    We call $\lambda$ a proposition if and only if $\lambda$ satisfies the following recurrence: 
        \begin{enumerate}
        \item 
        $\lambda = \top$ or $\lambda = \bot$ \\
        These are the base cases of the recursive definition of a proposition, and will always be propositions.
        \item 
        $\lambda = \neg \phi$, where $\phi$ is a proposition \\
        This definition only uses the $\neg$ connective, so it is sufficient for our proof.
        \item 
        $\lambda = \phi \land \psi$, where $\phi$ and $\psi$ are propositions. \\
        This definition only uses the $\land$ connective, so it is sufficient for our proof.
        \item 
        $\lambda = \phi \lor \psi$, where $\phi$ and $\psi$ are propositions. \\
        $\phi \lor \psi$ can be rewritten as $\neg \neg \phi \lor \neg \neg \psi$ by \emph{double negation}. We can then rewrite this as $\neg (\neg \phi \land \neg \psi)$ by \emph{De Morgan's Rules}. This definition is now sufficient for our proof since the only connectives it uses are $\neg$ and $\land$.
        \item
        $\lambda = \phi \rightarrow \psi$ \\
        $\phi \rightarrow \psi$  can be rewritten as $(\neg \phi \lor \psi)$ by \emph{conditional disintegration}. This can then be rewritten as $(\neg \neg \neg \phi \lor \neg \neg \psi)$ by \emph{double negation}. This can then be rewritten as $\neg (\neg \neg \phi \land \neg \psi)$ by \emph{De Morgan's Rules}. This definition is now sufficient for our proof since the only connectives it uses are $\neg$ and $\land$. 
        \item
        $\lambda = \phi \iff \psi$ \\
        This can be rewritten as $(\phi \rightarrow \psi) \land (\psi \rightarrow \phi)$ by the definition of a biconditional statement. This can then be rewritten as $(\neg \phi \lor \psi) \land (\neg \psi \lor \phi)$ by \emph{conditional disintegration}. This can then be rewritten as $(\neg \neg \neg \phi \land \neg \neg \psi) \land (\neg \neg \neg \psi \land \neg \neg \phi)$ by \emph{double negation}. This can then be rewritten as $\neg (\neg \neg \phi \land \neg \psi) \land \neg (\neg \neg \psi \land \neg \phi)$ by \emph{De Morgan's Rules}. This definition is now sufficient for our proof since the only connectives it uses are $\neg$ and $\land$. 

        Using this recursive definition, we can break down every proposition of finite length to only contain $\neg$ and $\land$ connectives. Infinitely decomposable propositions do not satisfy the definition of a proposition since they will never reach the base case. So therefore, we do not need to account for infinitely long statements. As such, this recursive definition is sufficient to express \emph{any} proposition, and $\neg$ and $\land$ can express every part of the recursive definition, so they can express \emph{any} proposition as a result.
        \begin{flushright}
            \textbf{Ergo.}
        \end{flushright}
        \end{enumerate}
    \item[(5 points) \quad 5.]
    Is there a \emph{single connective} capable of expressing \emph{any} proposition? Justify your answer with a proof.

    \emph{\textbf{Proof.}} Observe the following chain of reasoning. \\
    Since we proved that any proposition can be expressed with $\neg$ and $\land$ in problem 4, if we can create a logical connective that is able to express both connectives, we will then be able to express \emph{any} proposition with our single connective. To do so, I will create a new logical connective named ``anot'', which I will symbolize as $\bigotimes$ because why not. Anot's truth table looks as such: 
    \begin{center}
        \begin{tabular}{||c c c||}
            \hline
            $p$ & $q$ & $p \bigotimes q$ \\
            \hline \hline
            $\top$ & $\top$ & $\bot$ \\
            \hline
            $\top$ & $\bot$ & $\top$ \\
            \hline
            $\bot$ & $\top$ & $\top$ \\
            \hline
            $\bot$ & $\bot$ & $\top$ \\
            \hline
        \end{tabular}
    \end{center}    
    Essentially, $\bigotimes$ behaves like $\neg (p \land q)$. To prove that $\bigotimes$ can express any proposition though, we must prove that it can express $\neg$ and $\land$ individually.

    To express $\neg$ in terms of $\bigotimes$, we can use the expression:
    \begin{equation*}
        \neg p \equiv p \bigotimes p
    \end{equation*}
    These two expressions are logically equivalent:
    \begin{center}
        \begin{tabular}{||c c c||}
            \hline
            $p$ & $\neg p$ & $p \bigotimes p$ \\
            \hline \hline
            $\top$ & $\bot$ & $\bot$ \\
            \hline
            $\bot$ & $\top$ & $\top$ \\
            \hline
        \end{tabular}
    \end{center}
    To express $\land$ in terms of $\bigotimes$, we can use the expression:
    \begin{equation*}
        p \land q \equiv (p \bigotimes q) \bigotimes (p \bigotimes q)
    \end{equation*}
    These two expressions are logically equivalent as well:
    \begin{center}
        \begin{tabular}{||c c c c||}
            \hline
            $p$ & $q$ & $p \land q$ & $(p \bigotimes q) \bigotimes (p \bigotimes q)$ \\
            \hline \hline
            $\top$ & $\top$ & $\top$ & $\top$ \\
            \hline
            $\top$ & $\bot$ & $\bot$ & $\bot$ \\
            \hline
            $\bot$ & $\top$ & $\bot$ & $\bot$ \\
            \hline
            $\bot$ & $\bot$ & $\bot$ & $\bot$ \\
            \hline
        \end{tabular}
    \end{center}    
    Because $\neg$ and $\land$ are sufficient to express \emph{any} proposition and $\bigotimes$ can express $\neg$ and $\land$ individually, $\bigotimes$ can express \emph{any} proposition.
    \begin{flushright}
            \textbf{Ergo.}
        \end{flushright}
\end{enumerate}

\end{document}