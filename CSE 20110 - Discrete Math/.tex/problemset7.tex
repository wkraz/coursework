% Packages 
%--------------------

\documentclass[11pt,a4paper]{article}
\usepackage[utf8x]{inputenc}
\usepackage[T1]{fontenc}
\usepackage{amsthm,amsmath,amssymb}
\usepackage{xcolor}

\usepackage[pdftex]{graphicx} % Required for including pictures
\usepackage[pdftex,linkcolor=black,pdfborder={0 0 0}]{hyperref} % Format links for pdf
\usepackage{calc} % To reset the counter in the document after title page
\usepackage{enumitem} % Includes lists
\usepackage{ stmaryrd }
\frenchspacing % No double spacing between sentences
\linespread{1.2} % Set linespace
\usepackage[a4paper, lmargin=0.1666\paperwidth, rmargin=0.1666\paperwidth, tmargin=0.1111\paperheight, bmargin=0.1111\paperheight]{geometry} %margins
%\usepackage{parskip}

\usepackage[all]{nowidow} % Tries to remove widows
\usepackage[protrusion=true,expansion=true]{microtype} % Improves typography, load after fontpackage is selected
%---------------------------

\begin{document} 
\title{Discrete Math Problem Set 7}
\author{Will Krzastek}
\date{March 28th, 2024}
\maketitle

\begin{enumerate}

    % 1
    \item 
    Let $X$ be a set. Show that $(\forall Y \in \mathbb{P}(X))(|Y| \leq |X|) $. \\
    \emph{\textbf{Proof.}} Let $X, Y $ be sets. Assume $Y \in \mathbb{P}(X) $. Because $\mathbb{P}(X) $ is the set of all subsets of $X$,
    and $Y \in \mathbb{P}(X) $, we then know $Y \subseteq X $. By the definition of a subset, we know $\forall b(b \in Y \Rightarrow b \in X) $. \\
    Recall that $|Y| \leq |X| \Leftrightarrow \exists f (f: Y \hookrightarrow X)$ by the definition of \emph{equinumerosity}. \\
    Consider an $f$ where $f: Y \rightarrow X $. Also, let $a, b \in Y$. \\
    Now let $f$ be defined as $ f(a) = a $. Because $Y \subseteq X $, we know $\forall a (a \in Y \Rightarrow a \in X) $. \\
    Recall that the definition of injectivity is $\exists f(f(a) = f(b) \Rightarrow a = b) $. To prove $f$ is injective, assume $f(a) = f(b) $.
    Then by the definition of $f$, we know that $f(a) = a $ and $f(b) = b $. 
    So, we have $f(a) = a $ and $f(b) = b $ by the definition of $f$. \\
    Since $f(a) = f(b)$, we know $a = b$.  \\
    So, by the definition of \emph{injectivity}, we know that $f$ is injective. \\ 
    So, $|Y| \leq |X| $ by the definition of injectivity.
    \begin{flushright}
        $\blacksquare$
    \end{flushright}

    % 2
    \item 
    Show that $\forall X \forall Y (|X| \leq |Y| \Rightarrow \exists Z (Z \subseteq Y \land |X| = |Z|)) $. \\
    \textbf{\emph{Proof.}} Let $X, Y, Z$ be sets. Assume $|X| \leq |Y| $. By definition, this means that $\exists f(f: X \hookrightarrow Y) $, which means
    there exists an injective function $f$ from $X$ to $Y$. \\
    Now, let $Z := \{f(a) \mid a \in X \} $. Because $Y$ is the codomain of $f$, $Z $ only contains elements of $Y$, so $Z \subseteq Y $. \\
    Now, let $g : X \rightarrow Z  $ be the function $g(a) := f(a) $ where $a \in X $. \\
    Now, we want to show that $|X| = |Z| $, so we want to show that $g  $ is a bijection. To do so, we will independently show that $g $ is both an injection and a surjection. 
    First, we will show $g$ is injective. Recall the definition of injectivity: $(\forall a, b \in X)(g(a) = g(b) \Rightarrow a = b) $. Assume $g(a) = g(b) $. 
    By the definition of $g$, we then know that $f(a) = f(b) $. So, we have $a = b$ because $f$ is injective. So, $g$ is injective. \\
    Now, we want to show that $g$ is surjective. Let $h$ be an element of $Z$. Recall the definition of surjectivity: 
    $(\forall h \in Z)(\exists x \in X)(g(x) = h) $. $h = f(a) $ where $a \in X $. Thus, $g(a) = f(a) $ because we know this is true $\forall h \in Z $.      
    So, $g(a) = h $ by definition. Thus, $g $ is surjective. \\
    Because $g $ is injective \emph{and} surjective, $g$ is by definition bijective. Thus, we obtain $|X| = |Z| $ by definition. 
    Therefore, $|X| \leq |Y| \Rightarrow \exists Z(Z \subseteq Y \land |X| = |Z|) $.
    \begin{flushright}
        $\blacksquare$
    \end{flushright}

    % 3
    \item 
    Let $X, Y, Z $ be sets and consider $f : X \rightarrow Y $ and $g : Y \rightarrow Z $. 
    We define the \emph{composition} of $f$ with $g$ to be the function $g \circ f : X \rightarrow Z$ given by 
    $(g \circ f)(x) := g(f(x)) $ for all $x \in X$.
    \begin{enumerate}
        % 3a 
        \item 
        Show that, if $f$ and $g$ are both injections, then $g \circ f $ is injective.  \\
        \emph{\textbf{Proof.}} Let $X, Y, Z $ be sets. Let $f: X \rightarrow Y$ and $g: Y \rightarrow Z $. Recall $(\forall x \in X)(g \circ f (x):= g(f(x)) $).
        Also recall the definition of injectivity: $(\forall a, b \in X)(g(f(a)) = g(f(b)) \Rightarrow a  = b) $.
        Assume $f$ and $g$ are both injections. Also assume $g(f(a)) = g(f(b)) $. \\
        Because $g$ is injective, we know $f(a) = f(b) $. \\
        Because $f$ is injective, we then know $a = b$. \\
        So, if $g$ and $f$ are injections, $g \circ f $ is injective.  
        \begin{flushright}
            $\blacksquare$
        \end{flushright}

        % 3b 
        \item 
        Show that, if $f$ and $g$ are both surjections, then $g \circ f $ is surjective. \\
        \emph{\textbf{Proof.}} Let $X, Y, Z $ be sets. Let $f: X \rightarrow Y$ and $g: Y \rightarrow Z $. Recall the definition of surjectivity: $(\forall a \in A)(\exists b \in B)(f(b) = a) $. \\
        Assume $g $ and $f$ are both surjections. \\
        Because $f$ is surjective, $(\forall y \in Y)(\exists x \in X)(f(x) = y) $. \\
        Because $g$ is surjective, $(\forall z \in Z)(\exists y \in Y)(g(y) = z) $. \\
        Thus, $g(f(x)) = g(y) = z $ for arbitrary values of $y$ and $z$. \\
        So, if $g $ and $f$ are surjections, then $g \circ f  $ is a surjection.
        \begin{flushright}
            $\blacksquare$
        \end{flushright}

        % 3c
        \item 
        Show that, if $f$ and $g$ are both bijections, then $g \circ f $ is bijective.  \\
        \emph{\textbf{Proof.}} Let $X, Y, Z $ be sets. Let $f: X \rightarrow Y$ and $g: Y \rightarrow Z $. Recall the definition of bijectivity is possessing sujrectivity and injectivity. \\
        Assume $g$ and $f$ are bijections. \\
        So, $g$ and $f$ are both injective and surjective by definition. \\
        In \emph{3(a)}, we proved $g \circ f $ is injective when $g$ and $f$ are injective. \\
        In \emph{3(b)}, we proved $g \circ f $ is surjective when $g$ and $f$ are surjective. \\
        Because $g \circ f $ is injective \emph{and} surjective when $g$ and $f$ are bijections, $g \circ f $ is bijective.
        \begin{flushright}
            $\blacksquare$
        \end{flushright}

    \end{enumerate}
    \newpage
    
    % 4
    \item 
    For this problem, let $X$ and $Y$ be nonempty sets and let $f: X \rightarrow Y $.
    \begin{enumerate}
        %4a 
        \item 
        If $f$ is injective, show there exists $g: Y \rightarrow X  $ where $g \circ f = id_X $. \\
        \emph{\textbf{Proof.}} Let $X, Y $ be sets. Let $f : X \rightarrow Y $. Assume $f$ is injective.
        We want to show $(\exists g: Y \rightarrow X)(g \circ f = id_X) $.
        Because $X$ is nonempty, an arbitrary element $x \in X$ exists. Remember $id_X := g(f(x)) = x $. \\
        Because $f$ is injective, we know $f(x) = y$ for some distinct $y \in Y $, where $x$ is the only input mapped to $y$.
        Now, consider $g(y) := x $ where $y$ is the same $y$ as the output of $f(x)$. Here, we see that $g(f(x)) = g(y) = x $.  When $f(x) = y $, this would mean that $g(f(x)) $ = $g(y) $ = x. 
        Therefore, $g \circ f = x $, which means that $(\exists g: Y \rightarrow X)(g \circ f = id_X) $.
        \begin{flushright}
            $\blacksquare$
        \end{flushright}

        %4b 
        \item 
        If $f$ is surjective, show there exists $g: Y \rightarrow X $ where $f \circ g = id_Y $. \\
        \emph{\textbf{Proof.}} Let $X, Y$ be sets. Let $f: X \rightarrow Y $. Assume $f$ is surjective. 
        We want to show $(\exists g: Y \rightarrow X)(f \circ g = id_Y) $. Because $Y$ is nonempty, an arbitrary $y \in Y$ exists.
        Remember $id_Y := f(g(x)) = y $. \\
        Because $f$ is surjective, we know that $f(x) = y $ for some $x \in X$. Now, consider $g(y) := x $ where $x$ is the same $x$ that is input into $f$.
        This means that $f(g(y)) = f(x) = y $. 
        Therefore, $f \circ g = y $, which means that $(\exists g: Y \rightarrow X)(f \circ g = id_Y) $.
        \begin{flushright}
            $\blacksquare$
        \end{flushright}
        
        %4c
        \item 
        If $f$ is a bijection, then show there exists a function $g: Y \rightarrow X  $
        such that $g \circ f = id_X $ and $f \circ g = id_Y $. \\
        \emph{\textbf{Proof.}} Let $X, Y$ be sets. Let $f: X \rightarrow Y $. Assume $f$ is a bijection. 
        By definition, this means that $f$ is both surjective and injective. \\
        In \emph{4(a)}, we proved that when $f$ is an injection, there exists a function $g: Y \rightarrow X$ where $g \circ f = id_X $ and we defined this function as $g := x $. \\
        In \emph{4(b)}, we proved that when $f$ is a surjection, there exists a function $g: Y \rightarrow X $ where $f \circ g = id_Y $ and we defined this function as $g := x $. \\
        Therefore, when $f$ is injective and surjective, the function $g := x $ exists where $g \circ f = id_X $ and $f \circ g = id_Y $.
        \begin{flushright}
            $\blacksquare$
        \end{flushright}

    \end{enumerate}
    \newpage

    % 5
    \item 
    \emph{Euler's totient function} is the function: $\varphi_e: \mathbb{N} \rightarrow \mathbb{N} $
    that counts how many positive integers are \emph{coprime} with each $n \in \mathbb{N} $, defined below:
    \begin{equation*}
        \varphi_e (n) := |\{z \in \mathbb{N} \mid 1 \leq z \leq n \land gcd(z, n) = 1 \} |
    \end{equation*}
    \begin{enumerate}
        % 5a
        \item 
        If $p, k, m \in \mathbb{N}_+ $ are \emph{positive} naturals with $p$ prime and $m \leq p^k  $,
        then prove that gcd($p^k, m ) \neq 1 \Leftrightarrow p \mid m $. \\
        \emph{\textbf{Proof.}} Let $p, k, m \in \mathbb{N}_+ $ and assume $p$ is prime and $m \leq p^k $. We will prove this by cases. \\
        \emph{Case 1:} gcd($p^k, m $) $\neq 1 \Rightarrow p \mid m$. \\
        Assume gcd($p^k, m $) $\neq 1 $. This means that $p^k$ and $m$ share a common divisor. Recall that $p^k := p * p$ ($k$ many times). Thus, $p$ is the only prime factor of $p^k$ because of the unique prime factorization of $p^k$ and the FTA.
        Because they share a common divisor and $p$ is the only prime divisor of $p^k$, $p$ must divide $m$. Therefore, $p \mid m $. \\
        \emph{Case 2:} $p \mid m \Rightarrow gcd(p^k, m) \neq 1 $. \\
        Assume $p \mid m $. So, $p $ is a divisor of $m$. 
        Remember, $p$ is the only prime divisor of $p^k$ as shown in Case 1. As such $p$ is a divisor of $p^k$ and $m$, so gcd($p^k, m ) \geq p$, and $p$ is prime so $p > 1$. 
        Therefore, gcd($p^k, m ) \neq 1$. \\
        Thus, both cases hold so gcd$(p^k, m) \neq 1 \Leftrightarrow  p \mid m$. 
        \begin{flushright}
            $\blacksquare$
        \end{flushright}

        % 5b
        \item 
        If $p$ is prime, then prove that $\varphi_e (p) = p - 1 $. \\
        \emph{\textbf{Proof.}} Let $p \in \mathbb{N}_+$ and assume $p$ is prime. \\
        We then define $\varphi_e(p) := |\{z \in \mathbb{N} \mid 1 \leq z \leq p \land gcd(z, p) = 1 \}|  $. \\
        We want to now show that $\varphi_e(p) = p - 1 $. \\
        The set of all numbers that satisfies $1 \leq z \leq p $ is $[p + 1]$, but 0 is excluded, so its cardinality will be $p$. 
        Because $p$ is prime, the set of all numbers that satisfies $gcd (z, p) = 1 $ while being less than or equal to $p$ is every number strictly less than $p$. So, $\varphi_e(p)  $ will be $[p]$, but we still must exclude 0.
        So, $\varphi_e(p) = |[p] - 1 | $. A set with $p$ elements has cardinality $p$, so we equivalently get $\varphi_e(p) = p - 1 $.
        \begin{flushright}
            $\blacksquare$
        \end{flushright}

        % 5c
        \item 
        If $p$ is prime and $k \in \mathbb{N}_+ $, then prove that $\varphi_e (p^k) = p^k - p^{k - 1}$. \\
        \emph{\textbf{Proof.}} Let $p, k \in \mathbb{N}_+$ and assume $p$ is prime. \\
        We then define $\varphi_e(p^k) := |\{z \in \mathbb{N} \mid 1 \leq z \leq p^k \land gcd(z, p^k) = 1 \}| $. \\
        We want to show that $\varphi_e(p^k) = p^k - p^{k - 1} $. \\
        The set of all numbers that satisfies $1 \leq z \leq p^k $ is [$p^k + 1$], but 0 is excluded, so its cardinality is $p^k$. \\
        Because $p$ is prime and $p$ is the only prime divisor of $p^k$ as we proved in $5(a)$, $z $ cannot equal a multiple of $p$, or else they would share a common divisor and gcd$(z, p^k) \neq 1$. 
        So, the set of all numbers that satisfies $1 \leq z \leq p^k \land gcd(z, p^k) = 1 $ is $p^k $ minus the set of all multiples of $p$ up to $p^k$. 
        Let $M := $ the set of all multiples of $p$ up to $p^k$ excluding $p^k$. We can find $|M| $ by looking at the number of elements of the original set ($p^k)$, and dividing it by the difference between the first element (0) and $p$, which is just $p$.
        So, we see that $|M| = p^k / p $. By basic arithmetic, we know $p^k / p = p^{k - 1} $. Thus, $|M| = p^{k - 1} $. So, $\varphi_e(p^k) = p^k - |M| $, which is equal to $p^k - p^{k - 1} $.
        Therefore, $\varphi_e(p^k) = p^k - p^{k - 1} $.  
        \begin{flushright}
            $\blacksquare$
        \end{flushright} 

    \end{enumerate}
\end{enumerate}

\end{document}