% Packages 
%--------------------

\documentclass[11pt,a4paper]{article}
\usepackage[utf8x]{inputenc}
\usepackage[T1]{fontenc}
\usepackage{amsthm,amsmath,amssymb}
\usepackage{xcolor}
\usepackage{siunitx} % Scientific notation

\usepackage[pdftex]{graphicx} % Required for including pictures
\usepackage[pdftex,linkcolor=black,pdfborder={0 0 0}]{hyperref} % Format links for pdf
\usepackage{calc} % To reset the counter in the document after title page
\usepackage{enumitem} % Includes lists
\usepackage{ stmaryrd }
\frenchspacing % No double spacing between sentences
\linespread{1.2} % Set linespace
\usepackage[a4paper, lmargin=0.1666\paperwidth, rmargin=0.1666\paperwidth, tmargin=0.1111\paperheight, bmargin=0.1111\paperheight]{geometry} %margins
%\usepackage{parskip}

\usepackage[all]{nowidow} % Tries to remove widows
\usepackage[protrusion=true,expansion=true]{microtype} % Improves typography, load after fontpackage is selected
%---------------------------

\begin{document} 
\title{Discrete Math Problem Set 8}
\author{Will Krzastek}
\date{April 4th, 2024}
\maketitle

\begin{enumerate}
    % 1
    \item 
    Including the instructor, there are 32 people in our class. Prove that two of these people were born on the same day of the month. \\
    \emph{\textbf{Proof.}} Let $P := $ ``The set of 32 people in our class'' where $|P| = 32 $. \\
    Let $B :=  \{1, 2, \dots 31 \}$ where $B$ is every possible birthdate. Recall, the largest month in the year has 31 days, so $|B| = 31 $. \\
    Now, we know that $|P| > |B| $. \\ 
    Let $f: P \rightarrow B $ where $f(P_i) :=  $ ``the birthdate $B_j $ of $P_i $  ''. \\ 
    By the \emph{Pigeonhole Principle}, we know that $(\forall f: P \rightarrow B)$ $f$ is not injective. 
    So by definition, there exists $P_i, P_j \in P$ such that $P_i \neq P_j \land f(P_i) = f(P_j) $. \\
    Therefore, 2 people in our class were born on the same day of the month.
    \begin{flushright}
        $\blacksquare$
    \end{flushright}

    % 2
    \item 
    As of the $28^{\text{th}} $ of March 2024, there are over 8.1 billion people living on Earth. 
    A person's heart will beat no more than  $\num{7e9} $ times over their lifespan. Show that there are two currently-living people on Earth whose hearts have beat the same amount of times. \\
    \emph{\textbf{Proof.}} Let $E := \{1, 2, \dots 8.1 \times 10^9 \} $ where $E $ is the amount of people living on Earth. Observe, $|E| = 8.1 \times 10^9 $. \\
    Let $H := \{1, 2, \dots 7 \times 10^9 \} $ where $H$ is the amount of of heartbeats possible in a person's life. Observe, $|H| = 7 \times 10^9 $. \\
    Now, we know that $|E| > |H| $. \\
    Let $g: E \rightarrow H $ where $g(E_i) :=  $ ``the amount of heartbeats $H_j$ in the life of $E_i$''. \\
    By the \emph{Pigeonhole Principle}, we know that $(\forall g: E \rightarrow H) g $ is not injective. 
    So by definition, there exists $E_i, E_j \in E $ such that $E_i \neq E_j \land g(E_i) = g(E_j) $. \\
    Therefore, there are two people currently living on Earth whose hearts have beat the same amount of times. 
    \begin{flushright}
        $\blacksquare$
    \end{flushright}

    % 3
    \item 
    Let $n \in \mathbb{N}_+ $ and consider $A \subseteq \mathbb{N} $ such that $|A| = n + 1$. Prove that there exist $x, y \in A  $ with $x \neq y$ such that $n \mid x - y  $. \\
    \emph{\textbf{Proof.}} Let $A := \{a_1, a_2, \dots a_{n+1} \} $ where $|A| = n + 1 $ and $A \subseteq \mathbb{N} $. \\
    Let $R := \{0, 1, \dots n - 1 \} $ where $R$ is the set of possible remainders of any number $n$ divides. Observe, $|R| = n$. \\
    Now, we know that $|A| > |R| $. \\
    Now, let $f: A \rightarrow R $ where $f(A_i) :=  $ ``the remainder $R_j$ of $A_i$ divided by $n$''. \\
    By the \emph{Pigeonhole Principle}, we know that $(\forall f: A \rightarrow R) f $ is not injective. 
    So by definition, there exists $A_i, A_j \in A $ such that $A_i \neq A_j \land f(A_i) = f(A_j) $. \\
    Now, let $x := A_i $ and let $y := A_j $. So, we know that $x \neq y$ and the remainder of $n \mid x$ equals the remainder of $n \mid y$.
    We can then say $x = kn + r $ and $y = mn + r $ where $k, m \in \mathbb{N} $ and $r = r $. By arithmetic, we see that $x - y = (k - m)n  $.
    Since $k, m \in \mathbb{N} $, we know $n \mid x - y $. \\
    Therefore, there exist $x, y \in A $ where $x \neq y \land n \mid x - y$.
    \begin{flushright}
        $\blacksquare$
    \end{flushright}

    % 4
    \item 
    Consider $S := \{3, 4, 7, 8, 9, 10, 12, 15, 18, 19, 27, 28 \}  $ and $X \subseteq S$ with $|X| \geq 9 $.
    Show that there exists \emph{three} distinct elements $x_1, x_2, x_3 \in X$ such that $x_1 + x_2 + x_3 = 40 $.  \\
    \emph{\textbf{Proof.}} Consider $S := \{3, 4, 7, 8, 9, 10, 12, 15, 18, 19, 27, 28 \}  $. Observe, $|S| =  12$. Let $X \subseteq S  $ and $|X| \geq 9$. 
    We want to prove that there must exist $ x_1, x_2, x_3 \in X $ such that $x_1 + x_2 + x_3 = 40  $ and $x_1 \neq x_2 \neq x_3 $. \\
    Now, observe that there are 4 distinct sets of elements of $S$ that sum to 40: 
    \begin{align*}
        \{3, 10, 27 \} \\
        \{4, 8, 28 \} \\
        \{7, 15, 18 \} \\
        \{9, 12, 19 \}
    \end{align*}
    Now, let $P$ be defined as the set of all 4 of the above sets where:
    \begin{equation}
        P := \{ \{3, 10, 27 \},  \{4, 8, 28 \},  \{7, 15, 18 \}, \{9, 12, 19 \} \}.
    \end{equation}
    Observe, $|P| = 4 $. To show that there always exist $x_1, x_2, x_3  $ where $x_1 + x_2 + x_3 = 40 $,
    we simply need to prove an element of $P$ always exists. \\
    Now, observe $f: X \rightarrow P $ such that $f(x_i) := p_j $ where $x_i \in X $ and $p_j \in P$. \\
    By the \emph{Pigeonhole Principle}, we know there exists $p \in P $ such that $|\{x \in X \mid f(x) = p \} | \geq \lfloor \frac{9 - 1}{4} \rfloor + 1 = \lceil \frac{9}{4} \rceil $. \\
    We know that $\lceil \frac{9}{4} \rceil = 3 $ by the definition of the ceiling function.         \\ 
    So, $|p| = 3 $. Because $p \in P$, we then know that 3 distinct elements exist such that $x_1 + x_2 + x_3 = 40 \land x_1 \neq x_2 \neq x_3 $.
    \begin{flushright}
        $\blacksquare$
    \end{flushright}

    % 5 
    \item 
    Recall that $\binom{n}{0} = \binom{n}{n} = 1 $ for all $n, k \in \mathbb{N} $ when $k \leq n $.
    \begin{enumerate}

        % 5a
        \item 
        Show $\binom{n}{k} = \binom{n}{n - k} $ for all $n, k \in \mathbb{N} $ where $k \leq n $. \\
        \emph{\textbf{Proof.}} Assume $n, k \in \mathbb{N} $ and $k \leq n$. \\
        Let $A := \{z \mid z \subseteq n \land |z| = k \} $. \\
        Let $B := \{z \mid z \subseteq n \land |z| = n - k \} $. \\
        To show that $|A| = |B| $, we will prove the existence of a bijection from $A \rightarrow B $.  \\
        To do that, we will first show that there exists $f: A \rightarrow B  $ where $f$ is injective. \\
        To do that, let $f(a) := n \setminus a $. \\
        Assume $f(x) = f(y) $. This means that $n \setminus x = n \setminus y $. \\ 
        Towards a contradiction, suppose $x \neq y $. This means that there exists an element $b \in x $ where $b \notin y $. Because $x \subseteq n $, $b \in n $ by definition. 
        Because $b \in n  $ and $b \notin y $, $b \in n \setminus y $. We know that $n \setminus y = n \setminus x $, so $b \in n \setminus x $. This means that $b \notin x $. However, we assumed $b \in x \lightning$. \\
        Therefore, $x = y $, so $f  $ is injective. \\
        Now, we will show that there exists $g: B \rightarrow A $ where $g$ is injective. \\
        To do that, let $g(b) := n \setminus b $. \\
        Assume $g(x) = g(y) $. This means that $n \setminus x = n \setminus y $. \\ 
        Towards a contradiction, suppose $x \neq y $. This means that there exists an element $b \in x $ where $b \notin y $. Because $x \subseteq n $, $b \in n $ by definition. 
        Because $b \in n  $ and $b \notin y $, $b \in n \setminus y $. We know that $n \setminus y = n \setminus x $, so $b \in n \setminus x $. This means that $b \notin x $. However, we assumed $b \in x \lightning$. \\
        Therefore, $x = y $, so $g  $ is injective. \\
        Thus, there exist $f: A \rightarrow B $ and $g: B \rightarrow A $ where both are injective. \\
        Therefore, by the \emph{Cantor-Schroder-Bernstein} Theorem, we know there exists a bijection from $A \rightarrow B $. \\
        Therefore, $|A| = |B| $. \\
        By definition, we then know  $\binom{n}{k} = \binom{n}{n - k} $.
        \begin{flushright}
            $\blacksquare$
        \end{flushright} 
        
        % 5b
        \item 
        Show  $\binom{n + 1}{k + 1} = \binom{n}{k + 1} + \binom{n}{k} $ for all $n, k \in \mathbb{N} $ where $k \leq n $. \\
        \emph{\textbf{Proof.}} Assume $n, k \in \mathbb{N}  $ and $k \leq n $. \\
        Let $A := \{z \mid z \subseteq n + 1 \land |z| = k + 1 \} $. \\
        Let $B := \{z \mid z \subseteq n \land |z| = (k \lor k + 1) \} $. \\
        To show that $|A| = |B|$, we will show the existence of a bijection from $A \rightarrow B $. \\
        Observe,  $f: A \rightarrow B$ where $f(a) := a$ if $n \notin a $, and $f(a) := a \setminus n $ if $n \in a$. \\
        First, we will show that $f$ is injective. \\
        Assume $f(x) = f(y)$ where $x, y \in A $ . \\ 
        Look at the case where $|f(x) | = k + 1$. This also means that $f(y) = k + 1 $. \\
        Suppose towards a contradiction that $n \in x $. This means that $f(x) = x \setminus n $, so $|f(x)| = k $. However, we know $|f(x)| = k + 1 \lightning$. Thus, $n \notin x $. 
        So, $f(x) = x $ and $f(y) = y $. We know $f(x) = f(y) $, so $x = y$. \\
        Now, look at the case where $|f(x)| = k  $. Assume towards a contradiction that $n \notin x $. This means $|f(x)| = k + 1 $ as we showed in the first case. However, we assumed $|f(x)| = k \lightning$. 
        Therefore, we know $n \in x, y$. So, $f(x) = x \setminus n $ and $f(y) = y \setminus n $. Since we know $x \setminus n = y \setminus n $ and $n \in x, y $, we then know $x = y $. \\
        Therefore, $f$ is injective. \\
        Now, we will show that $f$ is surjective. \\
        Let $b \in B $. $|b| = k + 1 \lor k $ by the definition of $B$. \\
        First, look at the case $|b| = k + 1$. \\
        We know that $b \subseteq A $ and $|b| = k + 1 $. Therefore, we also know $b \in A$. Because $|b| = k + 1 $, we know $f(b) = b $ as we showed in the injective proof. 
        Therefore, there exists an input in $A$ for every $b \in B $ where $|b| = k + 1$. \\
        Now look at the case $|b| = k $. \\
        We know that $b \subseteq A  $ and $|b| = k $. Because $|b| = k $, we know $n \in b$. So, $b \cup n \in A $ and $f(b) = b \setminus n $. Therefore, there exists an input in $A$ for every $b \in B$ where $|b| = k $. \\
        Therefore, $f$ is surjective. \\
        Because we proved $f$ is injective and surjective, we know $f$ is bijective. So, $|A| = |B| $. \\
        We know that $|A| = \binom{n + 1}{k + 1} $. We can define $|B|$ as $|C \cup D |$ where $C := \{z \mid z \subseteq n \land |z| = k \} $ and $D := \{z \mid z \subseteq n \land |z| = k + 1 \} $. Observe, $|C \cap D| = 0 $ because $C$ and $D$ do not contain any of the same elements.
        Thus, $|C \cup D| = |C| + |D| $. We can define $|C|$ as $\binom{n}{k + 1} $ and $|D|$ as $\binom{n}{k} $. Therefore, $|B| = \binom{n}{k + 1} + \binom{n}{k} $. \\
        We know $|A| = |B| $, so $\binom{n + 1}{k + 1} = \binom{n}{k + 1} + \binom{n}{k} $.      
        \begin{flushright}
            $\blacksquare$
        \end{flushright} 

    \end{enumerate}

    % 6
    \item 
    Prove that $|\mathbb{P}(X) |  = 2^{|X|}$. \\
    \emph{\textbf{Proof.}} Let $X$ be a set. We will prove $|\mathbb{P}(X)| = 2^{|X|} $ this by induction on $|X| \in \mathbb{N} $. \\
    \emph{Basis step:} \\
    Observe, $|X| = 0 $, so $X := \varnothing  $. Therefore, $\mathbb{P}(X)  = \{\varnothing \} $, so $|\mathbb{P}(X)| = 1 $.
    We also know $2^{|\varnothing|} = 2^0 = 1 $. \\
    Therefore, $|\mathbb{P}(\varnothing)| = 2^{|\varnothing|} $. \\
    \emph{Inductive step:} \\
    Let $k \in \mathbb{N} $. Let $k := |X| $. Assume $|\mathbb{P}(X)| = 2^{|X|} $ as our inductive hypothesis. \\
    Now, let $m$ be a set such that $m \notin X$. Let $Y$ also be a set. Let $Y := X \cup m$. Now, observe that $|Y| = k + 1 $ because we just added one element to $X$.                      \\
    Now, we want to show that $|\mathbb{P}(Y)| = 2^{|Y|} $. \\
    Now, let's define a set $S$ that is every subset of $Y$ that contains $m$. \\ 
    $S := \{s \mid (\forall x \in \mathbb{P}(X))(s = x \cup m) \} $. Now, we see that $\mathbb{P}(Y) = \mathbb{P}(X) \cup S $,
    because $S$ is just every subset of $Y$ that contains $m$, and $\mathbb{P}(X) $ is all the subsets of $Y$ that do not. 
    Because they do not share any elements, we can say $|\mathbb{P}(Y)| = |\mathbb{P}(X)| + |S|$. We want to show that $|\mathbb{P}(X)| = |S| $.
    To do so, let's define a function $f: \mathbb{P}(X) \rightarrow S $ where $f(x) = x \cup m $. We will show that $f$ is bijective. \\
    First, let's show $f$ is injective. Let $x_1, x_2 \in \mathbb{P}(X) $. Assume $f(x_1) = f(x_2) $.
    This means that $x_1 \cup m = x_2 \cup m $. Since $m \notin \mathbb{P}(X) $, we know $a \notin x_1, x_2 $. So, we know $x_1 = x_2$. \\
    Now, we will show $f$ is surjective. Let $s \in S$. By the definition of $S$, $\exists x \in \mathbb{P}(X)  $ where $s = x \cup m$. So, every element of $S$ has an input value. Therefore $f$ is surjective. \\
    So, $f$ is bijective, and $|\mathbb{P}(X)| = |S| $. \\
    So, we can represent $|\mathbb{P}(X)| + |S|  $ as $2 \cdot |\mathbb{P}(X)| $. This means that $|\mathbb{P}(Y)| = 2 \cdot \mathbb{P}(X) $. By the \emph{Inductive hypothesis}, we know $2 \cdot |\mathbb{P}(X)| = 2 \cdot 2^{|X|} $. \\
    Observe, $2^{|Y|} =  2^{|X| + 1} = 2 \cdot 2^{|X|}$. \\
    Therefore, $|\mathbb{P}(Y)| = 2^{|Y|} $. \\
    Therefore, by \emph{mathematical induction}, we know $|\mathbb{P}(X) |  = 2^{|X|}$.
    \begin{flushright}
        $\blacksquare$
    \end{flushright} 
    
\end{enumerate}

\end{document}