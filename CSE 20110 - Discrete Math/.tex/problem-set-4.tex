%--------------------
% Packages
% -------------------
\documentclass[11pt,a4paper]{article}
\usepackage[utf8x]{inputenc}
\usepackage[T1]{fontenc}
\usepackage{amsthm,amsmath,amssymb}
\usepackage{xcolor}

\usepackage[pdftex]{graphicx} % Required for including pictures
\usepackage[pdftex,linkcolor=black,pdfborder={0 0 0}]{hyperref} % Format links for pdf
\usepackage{calc} % To reset the counter in the document after title page
\usepackage{enumitem} % Includes lists
\usepackage{ stmaryrd }
\frenchspacing % No double spacing between sentences
\linespread{1.2} % Set linespace
\usepackage[a4paper, lmargin=0.1666\paperwidth, rmargin=0.1666\paperwidth, tmargin=0.1111\paperheight, bmargin=0.1111\paperheight]{geometry} %margins
%\usepackage{parskip}

\usepackage[all]{nowidow} % Tries to remove widows
\usepackage[protrusion=true,expansion=true]{microtype} % Improves typography, load after fontpackage is selected

% Begin document
%-----------------------

\begin{document} 
\title{Discrete Math Problem Set 4}
\author{Will Krzastek}
\date{February 20, 2024}
\maketitle

\begin{enumerate}
    \item[(20 pts) \quad 1.]
    \begin{enumerate}
        % 1a
        \item 
        Show $\forall x (\varnothing \subseteq x)$. \\
        \emph{\textbf{Proof.}} Let $x$ be an arbitrary set. Towards a contradiction, suppose $\varnothing \nsubseteq x$. 
        Then there exists some $z$ such that $z \subseteq \varnothing \land z \nsubseteq x$ by definition. This claims that $z \in \varnothing$.
        However, we know $\forall y(y \notin \varnothing)$. $\lightning$. Therefore, $\varnothing \subseteq x$.
        \begin{flushright}
            $\blacksquare$
        \end{flushright}
        
        % 1b
        \item 
        Show $\forall x (x \subseteq x)$. \\
        \emph{\textbf{Proof.}} Let $x$ and $z$ be arbitrary sets. We know that $z \in x \Rightarrow z \in x$. Therefore by the definition of a subset, $x \subseteq x$. 
        \begin{flushright}
            $\blacksquare$
        \end{flushright}

        % 1c 
        \item 
        Show $\forall x (\varnothing \in \mathbb{P} (x))$. \\
        \emph{\textbf{Proof.}} Let $x$ be an arbitrary set. By Axiom 4, $\mathbb{P} (x) := \{z \mid z \subseteq x\}$. In \emph{$1a$}, we showed that $\forall x (\varnothing \subseteq x)$. 
        So, $\varnothing \in z$. Therefore, $\varnothing \in \mathbb{P} (x)$.
        \begin{flushright}
            $\blacksquare$
        \end{flushright}

        % 1d
        \item 
        Show $\forall x (x \in \mathbb{P} (x))$. \\
        \emph{\textbf{Proof.}} Let $x$ be an arbitrary set. By Axiom 4, $\mathbb{P} (x) := \{z \mid z \subseteq x\}$. In \emph{1(b)}, we showed that $\forall x (x \subseteq x)$. So, $x \in z$. Therefore, $x \in \mathbb{P} (x)$.
        \begin{flushright}
            $\blacksquare$
        \end{flushright}

        % 1e
        \item 
        Show $\forall x \forall y \forall z (((x \subseteq y) \land (y \subseteq z)) \Rightarrow x \subseteq z)$. \\
        \emph{\textbf{Proof.}} Let $x, y, z$ be arbitrary sets. To show $((x \subseteq y) \land (y \subseteq z)) \Rightarrow x \subseteq z$, assume $(x \subseteq y) \land (y \subseteq z)$. 
        Now suppose $a \in x$. Because $x \subseteq y$, we know $a \in y$. Because $y \subseteq z$, we know $a \in z$. 
        Because we chose an arbitrary element of $x$ and proved it must be an element of $z$, we can say $\forall s (s \in x \Rightarrow s \in z)$.
        By the definition of a subset, we can then say $x \subseteq z$. Because we derived $x \subseteq z$, from our assumption of $(x \subseteq y) \land (y \subseteq z)$, we can say $(x \subseteq y) \land (y \subseteq z) \Rightarrow x \subseteq z$.  
        \begin{flushright}
            $\blacksquare$
        \end{flushright}
    \end{enumerate}

    \item[(10 pts) \quad 2.]
    We define the \emph{intersection} and \emph{difference} of any two sets $x$ and $y$ below.
    \begin{equation*}
     x \cap y := \{ z \mid z \in x \land z \in y\} \\ 
    \end{equation*}
    \begin{equation*}
        x \backslash y := \{ z \mid z \in x \land z \notin y\} 
    \end{equation*}
    \begin{enumerate}

        % 2a
        \item 
        Show $\forall x \forall y \exists z (z = x \cap y)$. \\
        \emph{\textbf{Proof.}} Let $x, y, z$ be arbitrary sets. By \emph{Schema of Separation}, we know that $\forall x \exists z (z = \{t \mid t \in x \land \varphi (t)\})$.
        Now, let the predicate $\varphi (t) := t \in y$. We can then equivalently say $\forall x \forall y \exists z (z = \{t \mid t \in x \land t \in y\}). $
        By the definition of \emph{intersection}, we can then say $\forall x \forall y \exists z (z = x \cap y)$.
        \begin{flushright}
            $\blacksquare$
        \end{flushright}

        % 2b
        \item 
        Show $\forall x \forall y \exists z (z = x \backslash y)$. \\
        \emph{\textbf{Proof.}} Let $x, y, z$ be arbitrary sets. By \emph{Schema of Separation}, we know that $\forall x \exists z (z = \{t \mid t \in x \land \varphi (t) \} ).  $ 
        Now let the predicate $\varphi (t) := t \notin y  $. We can then equivalently say $\forall x \forall y \exists z (z = \{t \mid t \in x \land t \notin y \} ).  $ 
        By the definition of \emph{difference}, we can then say $\forall x \forall y \exists z (z = x \backslash y ).  $ 
        \begin{flushright}
            $\blacksquare$
        \end{flushright}
    \end{enumerate}
    \item[(20 pts) \quad 3.]
    We define the \emph{union} of two sets $x$ and $y$ below as: 
    \begin{equation*}
        x \cup y := \{z \mid z \in x \lor z \in y \}.
    \end{equation*}
    \begin{enumerate}
        
        % 3a
        \item 
        Show $\forall x \forall y (x \cap y \subseteq x)$. \\
        \emph{\textbf{Proof.}} Let $x, y$ be arbitrary sets. \\
        Suppose $a \in x \cap y$. We now know that $a = \varnothing \lor a \neq \varnothing$.
        In the case $a = \varnothing$, we know $\varnothing \subseteq x$ because we proved it in \emph{1a}.
        In the case $a \neq \varnothing$, we know $a \in x \land a \in y$ by the definition of \emph{intersection}. By \emph{conjunciton elimination}, we know $a \in x$. Because we showed this for an arbitrary element of $x \cap y$, we can say $\forall b (b \in x \cap y \Rightarrow b \in x)$.
        By the definition of a subset, we now obtain $x \cap y \subseteq x$. 
        \begin{flushright}
            $\blacksquare$
        \end{flushright}

        % 3b
        \item 
        Show $\forall x \forall y (x \subseteq x \cup y)$. \\
        \emph{\textbf{Proof.}} Let $x, y$ be arbitrary sets. \\
        Suppose $a \in x$. \\
        $x \cup y := \{b \mid  b \in x \lor b \in y \}$. Since we know $a \in x $, by \emph{disjunction introduction}, we can say $a \in x \lor a \in y$. By the definition of \emph{union} we then know $a \in x \cup y$.
        Because we derived $a \in x \cup y$ from our assumption $a \in x$. We can say $a \in x \Rightarrow a \in x \cup y$.
        Since we chose an arbitrary element of $x$, we can say $\forall c (c \in x \Rightarrow c \in x \cup y)$.
        This is the definition of a subset, so we obtain $x \subseteq x \cup y$.   
        \begin{flushright}
            $\blacksquare$
        \end{flushright}

        % 3c
        \item 
        Show $\forall x \forall y (\mathbb{P} (x) \cup \mathbb{P} (y) \subseteq \mathbb{P} (x \cup y))$. \\
        \emph{\textbf{Proof.}} Let $x, y$ be arbitrary sets. \\
        Suppose $a \subseteq x$ and separately suppose $b \subseteq y$. \\
        Because $\mathbb{P} (x)$ is defined as a set of all the subsets of $x$, we know that $a \in \mathbb{P} (x) $.
        Similarly, $\mathbb{P} (y) $ is defined as a set of all the subsets of $y$, so we know that $b \in \mathbb{P} (y) $.
        By \emph{2b} we know that $x \subseteq x \cup y$. Because $a \subseteq x$, we can say $a \subseteq x \cup y$.
        Because $b \subseteq y$, we can say $b \subseteq x \cup y$. Since $a$ and $b$ are both subsets of $x \cup y$, we can say $a \cup b \subseteq x \cup y$. 
        Because $\mathbb{P} (x \cup y)$ is the set of all the subsets of $x \cup y$ and $a \cup b$ is a subset of $x \cup y$, we can say $a \cup b \in \mathbb{P} (x \cup y)$.
        Because we derived $a \cup b \in \mathbb{P} (x \cup y)$ from $a \in \mathbb{P} (x) \cup b \in \mathbb{P} (y)$,
        we can say $a \in \mathbb{P} (x) \cup b \in \mathbb{P} (y) \Rightarrow a \cup b \in \mathbb{P} (x \cup y)$. Because we chose arbitrary elements of $x$ and $y$, 
        we can say $\forall c \forall d (c \in \mathbb{P} (x) \cup d \in \mathbb{P} (y) \Rightarrow c \cup d \in \mathbb{P} (x \cup y))$.
        This is the definition of a subset, so we obtain $\mathbb{P} (x) \cup \mathbb{P} (y) \subseteq \mathbb{P} (x \cup y)$.  
        \begin{flushright}
            $\blacksquare$
        \end{flushright}

        % 3d
        \item 
        Show $\forall x \forall y (x \cap y = x \Leftrightarrow x \in \mathbb{P} (y))$. \\
        \emph{\textbf{Proof.}} Let $x, y$ be arbitrary sets. \\
        To show $x \cap y = x \Leftrightarrow x \in \mathbb{P} (y)$, we need to show: (1) $x \cap y = x \Rightarrow x \in \mathbb{P} (y) $ and (2) $x \in \mathbb{P} (y) \Rightarrow x \cap y = x $
            \begin{enumerate}
                \item 
                We want to show that $x \cap y = x \Rightarrow x \in \mathbb{P} (y) $. \\
                Assume $x \cap y = x$. \\ 
                $x \cap y := \{z \mid z \in x \land z \in y\} $. By \emph{extensionality}, we then know $\forall z (z \in x \Leftrightarrow z \in x \cap y) $.  
                By the definition of \emph{intersection}, we can equivalently say $\forall z(z \in x \Leftrightarrow z \in x \land z \in y )$. Now assume $z \in x $. By \emph{modus ponens} we get $z \in x \land z \in y $. By \emph{conjunction elimination}, we get $z \in y $.
                Because $z \Leftrightarrow z \in x \land z \in y $, we know $z \in x \Rightarrow z \in x \land z \in y $. Because we chose an arbitrary $z$, and derived $z \in y $ from $z \in x $, so we can say $\forall h (h \in x \Rightarrow h \in y) $. 
                This is the definition of a subset, so we can say $x \subseteq y $. By the definition of a power set, we know $x \in \mathbb{P} (y) $. So, $x \cap y = x \Rightarrow x \in \mathbb{P} (y) $.   

                \item 
                We now want to show that $x \in \mathbb{P} (y) \Rightarrow x \cap y = x $. \\
                Assume $x \in \mathbb{P} (y) $. \\
                Because $x$ is in the power set of $y$, we know that $x \subseteq y $. $x \subseteq y := \{z \mid z \in x \Rightarrow z \in y \} $.
                So, we know that $z \in x \Rightarrow z \in y $. To show $x \cap y = x $, we want to show $z \in x \Leftrightarrow z \in x \land z \in y $.
                To do so, we will show (1) $z \in x \Rightarrow z \in x \land z \in y $ and (2) $z \in x \land z \in y \Rightarrow z \in x $.
                    \begin{enumerate}
                        \item[1.]
                        Assume $z \in x $. Because $x \subseteq y $, we know $z \in x \Rightarrow z \in y $. By \emph{conjunction introduction}, we then get $z \in x \Rightarrow z \in x \land z \in y $.
                        \item[2.]
                        Assume $z \in x \land z \in y $. By \emph{conjunction elimination}, we get $z \in x $. Since we derived $z \in x $ from our assumption, we can say $z \in x \land z \in y \Rightarrow z \in x $.    
                    \end{enumerate}
                So, $z \in x \Leftrightarrow z \in x \land z \in y $. Because this was an arbitrary $z$, we can say $\forall a (a \in x \Leftrightarrow a \in x \land a \in y) $. 
                By the definition of \emph{intersection}, we can say $\forall a (a \in x \Leftrightarrow a \in x \cap y) $. 
                By \emph{existentiality}, we can reduce this to $x = x \cap y $. So, $x \in \mathbb{P} (y) \Rightarrow x = x \cap y $.     
            \end{enumerate}
            Because we proved $x \cap y = x \Rightarrow x \in \mathbb{P} (y) $ and $x \in \mathbb{P} (y) \Rightarrow x \cap y = x $, we know $x \cap y = x \Leftrightarrow x \in \mathbb{P} (y)$.
            \begin{flushright}
                $\blacksquare$
            \end{flushright}
    \end{enumerate}
    \item[(50 pts) \quad 4.]
    We define the \emph{union over x} and \emph{intersection over x} for any set $x$ below. 
        \begin{equation*}
            \cup x := \{z \mid \exists y (y \in x \land z \in y) \}
        \end{equation*}
        \begin{equation*}
            \cap x := \{z \mid \forall y (y \in x \Rightarrow z \in y) \}
        \end{equation*}
    \begin{enumerate}
        
        % 4a 
        \item 
        Show that $\forall x (\cup \mathbb{P} (x) = x) $. \\
        \emph{\textbf{Proof.}} Let $x $ be an arbitrary set. \\
        By \emph{1(d)}, we know $x \in \mathbb{P}(x) $. We also know $\cup \mathbb{P}(x) := \{z \mid \exists y (y \in \mathbb{P}(x) \land z \in y) \}$.
        By \emph{existential elimination}, we can then say $\{z \mid x \in \mathbb{P}(x) \land z \in x \} $. 
        Since we know $x \in \mathbb{P}(x) $, by \emph{conjunction elimination} we can say $\{z \mid z \in x \}  $.
        Since all elements of $\{z \mid z \in x \}  $ are in $x$, and all elements of $x$ are in $\{z \mid z \in x \}  $, by \emph{existentiality} we can conclude $\{z \mid z \in x \} = x $.
        Since we determined $\{z \mid z \in x \} = \cup \mathbb{P}(x) $, we conclude $\cup \mathbb{P}(x) = x $.
        \begin{flushright}
            $\blacksquare$
        \end{flushright}

        % 4b
        \item 
        \emph{\textbf{Proof.}} What is $\cup \varnothing $? Justify your answer with a proof. \\
        We know that $\cup \varnothing := \{z \mid \exists y (y \in \varnothing \land z \in y) \} $. 
        However, by the definition of the empty set, we know there is no $y$ such that $y \in \varnothing $. As such, $\cup \varnothing  $ is an empty set because the predicate is false for every possible element of $\cup \varnothing $. 
        Therefore, $\cup \varnothing = \varnothing $.   
        \begin{flushright}
            $\blacksquare$
        \end{flushright}
        
        % 4c 
        \item 
        \emph{\textbf{Proof.}} What is $\cap \varnothing $? Justify your answer with a proof. \\
        We know that $\cap \varnothing := \{z \mid \forall y (y \in \varnothing \Rightarrow z \in y) \}  $. 
        By the definition of the empty set, we know that $\neg \exists y (y \in \varnothing) $. So, the consequent in the conditional statement is always false.
        Because the consequent of the conditional statement is false for all $y$, the conditional statement is then true for all $y$. So, we can conclude that every element exists in $\cap \varnothing$. Therefore, $\cap \varnothing \in \cap \varnothing $. 
        However by \emph{Russell's Paradox}, we know this to be false. Therefore, $\cap  \varnothing $ does not exist. 
        \begin{flushright}
            $\blacksquare$
        \end{flushright}

        % 4d
        \item 
        Is $\varnothing = \{z \mid z \in \varnothing \}  $ ? Justify your anwer with a proof. \\
        \emph{\textbf{Proof.}} Let $z$ be a set. By the definition of the empty set, we know $\neg \exists z (z \in \varnothing) $.
        So, there are no elements of $z$ that satisfy $z \in \varnothing $. 
        So, the set of all elements that satisfy $\{z \mid z \in \varnothing \} $ is empty because the empty set contains no elements. 
        Thus, $\varnothing = \{z \mid z \in \varnothing \} $.
        \begin{flushright}
            $\blacksquare$
        \end{flushright}
        
        % 4e 
        \item 
        Is $\varnothing = \{z \mid z \notin \varnothing \} $? Justify your answer with a proof. \\
        \emph{\textbf{Proof.}} Let $z$ be a set. By the definition of the empty set, we know $\neg \exists z (z \in \varnothing) $. By \emph{existential elimination}, we then know $\forall z (z \notin \varnothing) $. 
        Because the empty set cannot contain any elements, the set of all elements that satisfy $\{z \mid z \notin \varnothing \} $ is every element. Thus, $\varnothing \neq \{z \mid z \notin \varnothing \} $ because the empty set is not equal to a set with infinite elements. 
        \begin{flushright}
            $\blacksquare$
        \end{flushright}
    \end{enumerate}
        
\end{enumerate}



\end{document}